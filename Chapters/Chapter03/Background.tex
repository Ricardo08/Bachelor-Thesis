\section{Antecedentes}

Sea un haz principal $\pi: P \longrightarrow \mathcal{M}$ con grupo compacto de estructura $G$ y sea $S < \mathrm{Aut}(P)$ un grupo compacto actuando en el haz principal $P$, el cual, a trav�s del mapeo de proyecci�n $\pi$, induce una S-acci�n en la base $\mathcal{M}$. Dado que el grupo de simetr�a es compacto existe una subvariedad densa y abierta $\mathcal{M}_{(J)} \subset \mathcal{M}$, que localmente est� foliada por las �rbitas de $S$. Esto es, $\mathcal{M}_{(J)} \cong  \mathcal{M}_{(J)}/S \times S/J$ para un subgrupo compacto $J \subset S$. Con una peque�a p�rdida de generalidad, podemos asumir que el espacio base $\mathcal{M}$ se puede descomponer de la siguiente forma, $$\mathcal{M} = \mathcal{\tilde{M}} \times S/J,$$ donde $\mathcal{\tilde{M}} \cong \mathcal{M}/S$ es una variedad conexa, y la acci�n de $S$ en $\mathcal{M}$ es tal que las �rbitas est�n dadas por $S(m) \cong S/J$ para todo $m \in \mathcal{M}$.

Si la acci�n inducida de $S$ en $\mathcal{M}$ es transitiva pero no libre, isotrop�as adicionales ocurren. En cada punto $m \in \mathcal{M}$ se tiene un subgrupo de isotrop�a no trivial $J_{m}$ que deja fijo a $m$. Para puntos distintos $m$ y $m'$, donde por transitividad existe al menos un $s \in S$ tal que $m' = s(m)$, entonces $J_{m}$ s�lo difiere por conjugaci�n de $J_{m'}$,
%
\begin{equation*}
m' = j'(m') = s ( j ( s^{-1} (m') ) )  \quad \Longrightarrow j' = s \circ j \circ s^{-1} \qquad \forall \; j \in J_{m} \; \land \; \forall \; j' \in J_{m'},
\end{equation*}
%
lo que permite escoger alg�n $J$ para un punto arbitrario en $\mathcal{M}$. De esta manera se puede identificar a $\mathcal{M}$ con $S/J$.

Aunque la acci�n de $J$ en $\mathcal{M}$ es trivial no necesariamente lo es en a fibra $\pi^{-1} (m)$, lo que define $J: \pi^{-1} (m) \longrightarrow \pi^{-1} (m)$. Y como la acci�n derecha de $G$ en el haz es transitiva, a cada punto $p \in \pi^{-1}(m)$ en la fibra es posible asignar un homomorfismo que va del subgrupo de isotrop�a $J$ al grupo de estructura $G$ definido por $p \cdot j = p \cdot \lambda_{p} (j)$. Dado $p' = p \cdot g$ se tiene
%
\begin{equation*}
p' \cdot \lambda_{p'} (j) = j \cdot (p \cdot g) = (j \cdot p) \cdot g = (p \cdot \lambda_{p} (j)) \cdot g = p' \cdot \mathrm{Ad}_{g^{-1}} (\lambda_{p} (j)),
\end{equation*}
%
lo que implica que $\lambda_{p'} (j) = \mathrm{Ad}_{g^{-1}} (\lambda_{p} (j))$. Con esto se muestra la propiedad de homomorfismo,
%
\begin{equation*}
p \cdot \lambda_{p} (j_{1} j_{2}) = (j_{1} j_{2}) \cdot p = j_{1} \cdot (p \cdot \lambda_{p} (j_{2})) = (p \cdot \lambda_{p} (j_{2})) \cdot \mathrm{Ad}_{\lambda(j_{2})^{-1}} \lambda_{p} (j_{1}) = p \cdot (\lambda_{p} (j_{1}) \lambda_{p} (j_{2})).
\end{equation*}
%
De esta manera, un elemento que ayuda a la clasificaci�n del haz principal es una clase de equivalencia $[\lambda]$ de homomorfismos $\lambda: J \longrightarrow G$ sujetos a conjugaci�n. Donde cambiar $\lambda$ dentro de la clase equivale a un cambio de norma. El homomorfismo del �lgebra de Lie inducido $\mathfrak{j} \longrightarrow \mathfrak{g}$ tambi�n ser� denotado por $\lambda$.

Sea $\omega$ una conexi�n invariante en el haz principal $P$ bajo la acci�n de $S$, i.e. $s^{*}(\omega) = \omega$ para todo $s \in S$. Definimos un mapeo lineal $\Lambda: \mathfrak{s} \longrightarrow \mathfrak{g}$ por
%
\begin{equation}
\label{eq:correspondence}
\Lambda (X) = \omega (\tilde{X}), \qquad X \in \mathfrak{s},
\end{equation}
%
donde $\tilde{X}$ es el campo vectorial en $P$ inducido por $X$. As�,
%
\begin{align}
\label{eq:LambdaCond1}
& \Lambda (X) = \lambda (X) & \mathrm{para} \; X\in \mathfrak{j}; \\
\label{eq:LambdaCond2}
& \Lambda (\mathrm{ad}_{j} (X)) = \mathrm{ad}_{\lambda (j)} (\Lambda (X)) \qquad & \mathrm{para} \; j \in J \; \land X \in \mathfrak{s},
\end{align}
%
donde $\mathrm{ad}_{j}$ es la representaci�n adjunta de $J$ en $\mathfrak{s}$ y $\mathrm{ad}_{\lambda (j)}$ la representaci�n adjunta de $G$ en $\mathfrak{g}$.

\begin{thm}
\label{thm:theorem1}
Si el grupo de Lie $S < \mathrm{Aut} (P)$ es transitivo y si $J$ es el grupo de isotrop�a de $S$ en $m = \pi(p)$, entonces existe una correspondencia 1:1 entre el conjunto de conexiones invariantes en $P$ bajo la acci�n de $S$ y el conjunto de mapeos lineales $\Lambda: \mathfrak{s} \longrightarrow \mathfrak{g}$, el cual satisface las condiciones \eqref{eq:LambdaCond1} y \eqref{eq:LambdaCond2}. La correspondecia est� dada por \eqref{eq:correspondence}.
\end{thm}

\begin{thm}
Si $S$, en el teorema \ref{thm:theorem1}, admite un subespacio $\mathfrak{j}_{\perp}$ tal que $\mathfrak{s} = \mathfrak{j} \oplus \mathfrak{j}_{\perp}$ y (abusando un poco de la notaci�n) $\mathrm{ad}_{J} (\mathfrak{j}_{\perp}) = \mathfrak{j}_{\perp}$. Entonces existe una correspondecia 1:1 entre el conjunto de conexiones invariantes en $P$ bajo la acci�n de $S$ y el conjunto de mapeos lineales $\Lambda_{\mathfrak{j}_{\perp}}: \mathfrak{j}_{\perp} \longrightarrow \mathfrak{g}$ tal que
%
\begin{equation}
\label{eq:HiggsField}
\Lambda_{\mathfrak{j}_{\perp}} (\mathrm{ad}_{j} (X)) = \mathrm{ad}_{\lambda (j)} (\Lambda_{\mathfrak{j}_{\perp}} (X)),
\end{equation}
%
para $j \in J$ y $X \in \mathfrak{j}_{\perp}$. Con la correspondencia dada por \eqref{eq:correspondence}.
%
$$
\Lambda (X) = 
\begin{cases}
\lambda (X) \qquad & \mathrm{si} \; X\in \mathfrak{j}; \\
\Lambda_{\mathfrak{j}_{\perp}} (X) \qquad & \mathrm{si} \; X \in \mathfrak{j}_{\perp}.
\end{cases}
$$
%
\end{thm}

De la descomposici�n $\mathcal{M} = \mathcal{\tilde{M}} \times S/J$ se tiene
%
\begin{equation}
\omega = \tilde{\omega} + \omega_{S/J},
\end{equation}
%
donde $\tilde{\omega}$ es una conexi�n en el subhaz principal reducido $\tilde{Q}(\tilde{\mathcal{M}}, Z_{\lambda}, \tilde{\pi}) = \{p \in P \vert_{\tilde{\mathcal{M}}} \vert \lambda_{p} = \lambda\}$, con $Z_{\lambda}$ el centralizador de $\lambda(J) \subset G$, y $\omega_{S/J}$ es una uno-forma invariante en $S/J$. Sin embargo, en este trabajo $\tilde{\omega}$ no nos concierne pues al fijar un punto $m \in \mathcal{M}$, adem�s de fijar un grupo de isotrop�a $J$ (sujeto a conjugaci�n) y una clase de equivalencia $\lambda$, tambi�n ocurre que el espacio base $\tilde{\mathcal{M}} \cong \{[m]\}$ por lo que en realidad no existe una definici�n de $\tilde{\omega}$ as� que toda la informaci�n de $\omega$ est� contenida en $\omega_{S/J}$. Por otro lado, $S/J$ no necesariamente es un grupo, por lo que no hay una forma de Maurer-Cartan definida para usar para conexiones invariantes, pero se puede obtener $\omega_{S/J}$ usando la forma de Maurer-Cartan $\theta_{(\mathrm{MC})}$ en $S$ y escogiendo un encaje $\iota: S/J \hookrightarrow S$, expl�citamente:
%
\begin{equation*}
\omega_{S/J} (\tilde{X}) = \Lambda (X) = \Lambda (\iota^{*} \theta_{\mathrm{MC}} (\tilde{X})) = \Lambda \circ \iota^{*} \theta_{\mathrm{MC}} (\tilde{X})
\end{equation*}
%
de este modo se puede escribir:
%
\begin{equation}
\omega_{S/J} = \Lambda \circ \iota^{*} \theta_{\mathrm{MC}}.
\end{equation}