\section{Antecedentes}

\subsection{Clasificaci�n de Haces Principales sim�tricos}

Sea un haz principal $\pi: P \longrightarrow \mathcal{M}$ con grupo de estructura compacto $G$ y sea $S < \mathrm{Aut}(P)$ un grupo compacto de simetr�a actuando en el haz principal $P$, el cual, a trav�s del mapeo de proyecci�n $\pi$, induce una S-acci�n en la variedad base $\mathcal{M}$. Dado que el grupo de simetr�a es compacto existe una subvariedad densa y abierta $\mathcal{M}_{(K)} \subset \mathcal{M}$, que localmente est� foliada por las �rbitas de $S$. Esto es, $\mathcal{M}_{(K)} \cong  \mathcal{M}_{(K)}/S \times S/K$ para un subgrupo compacto $K \subset S$. El espacio base $\mathcal{M}$ se puede descomponer de la siguiente forma, $$\mathcal{M} = \mathcal{\tilde{M} \times S/K},$$ donde $\mathcal{\tilde{M}$ es una variedad conexa, y la acci�n de $S$ en $\mathcal{M}$ es tal que las �rbitas est�n dadas por $S(m) \cong S/K$ para todo $m \in \mathcal{M}$.

\subsection{Clasificaci�n de Conexiones invariantes}