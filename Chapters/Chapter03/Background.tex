Muchos de los fen\'{o}menos pronosticados por la relatividad general se entienden y analizan mejor mediante reducciones f\'{i}sicamente motivadas a situaciones m\'{a}s simples. Una de las principales t\'{e}cnicas es la reducci\'{o}n de simetr\'{i}a, es decir, las soluciones que presentan ciertas propiedades de simetr\'{i}a espacio-temporal. Los ejemplos que nosotros estudiaremos son las geometr\'{i}as espacialmente homog\'{e}neas e isotr\'{o}picas para modelos cosmol\'{o}gicos.

En cada uno de los modelos sim\'{e}tricos, homog\'{e}neos e isotr\'{o}picos, existe cierto grupo de simetr\'{i}a $S$ que act\'{u}a sobre el espacio-tiempo junto con una funci\'{o}n de tiempo $t$ tal que $S$ act\'{u}a transitivamente sobre cada corte espacial $\Sigma_{t}$, esto es, para cualquier par $(y, y')$ de puntos en $\Sigma_{t}$, i.e. existe un elemento de grupo $s \in S$ tal que $s(y) = y'$. Adem\'{a}s, en $\Sigma_{t}$  la m\'{e}trica inducida $h_{ab}$ y la curvatura extr\'{i}nseca $K_{ab}$ se preservan bajo la acci\'{o}n de $S$, es decir, $s^{*}(h_{ab}) = h_{ab}$ y $s^{*}(K_{ab}) = K_{ab}$ para todo $s \in S$. Con el formalismo presentado por Ashtekar \cite{Ashtekar86, Ashtekar87} y los estudios hechos por Kobayashi, Nomizu y Brodbeck sobre la clasificaci\'{o}n de haces principales sim\'{e}tricos y conexiones invariantes \cite{Kobayashi, Brodbeck} podemos encontrar la forma de las Variables de Conexi\'{o}n para estos modelos cosmol\'{o}gicos. Esto nos da un esquema general de clasificaci\'{o}n de todas las posibles formas invariantes para estos tipos de simetr\'{i}as ya que a partir de conexiones invariantes se puede encontrar la forma de m\'{e}tricas invariantes (ver \autoref{app:FrameField}).

A continuaci\'{o}n se presentar\'{a} c\'{o}mo con los estudios de clasificaci\'{o}n podemos encontrar la forma de las Variables de Conexi\'{o}n para los modelos cosmol\'{o}gicos homog\'{e}neos e isotr\'{o}picos.

\section{Antecedentes}

Sea un haz principal $\pi: P \longrightarrow \Sigma$ con un grupo compacto de Lie $G$ actuando por la derecha en $P$, es decir, $G$ es el grupo de estructura. Proponemos un subgrupo de Lie de simetr\'{i}a $S < \mathrm{Aut}(P)$ actuando en el haz principal $P$, el cual, a trav\'{e}s del mapeo de proyecci\'{o}n $\pi$, induce una $S$-acci\'{o}n en la base $\Sigma$. Dado que el grupo de simetr\'{i}a es compacto existe una subvariedad densa y abierta $\Sigma_{(J)} \subset \Sigma$, que al menos localmente est\'{a} foliada por las \'{o}rbitas de $S$. Esto es, $\Sigma_{(J)} \cong  \Sigma_{(J)}/S \times S/J$ para un subgrupo compacto $J < S$ del grupo de simetr\'{i}a \cite{Dieck}. Con una peque\~{n}a p\'{e}rdida de generalidad, podemos asumir que el espacio base $\Sigma$ se puede descomponer de la siguiente forma, $$\Sigma = \tilde{\Sigma} \times S/J,$$ donde $\tilde{\Sigma} \cong \Sigma/S$ es una variedad conexa, y la acci\'{o}n de $S$ en $\Sigma$ es tal que las \'{o}rbitas est\'{a}n dadas por $S(y) \cong S/J$ para todo $y \in \Sigma$.

Si la acci\'{o}n inducida de $S$ en $\Sigma$ es transitiva pero no libre, isotrop\'{i}as adicionales a la trivial ocurren. En cada punto $y \in \Sigma$ se tiene un subgrupo de isotrop\'{i}a no trivial $J_{y}$ que deja fijo a $y$. Para puntos distintos $y$ y $y'$, donde por transitividad existe al menos un $s \in S$ tal que $y' = s(y)$, entonces $J_{y}$ s\'{o}lo difiere por conjugaci\'{o}n de $J_{y'}$,
%
\begin{equation*}
y' = j'(y') = s ( j ( s^{-1} (y') ) )  \quad \Longrightarrow j' = s \circ j \circ s^{-1} \qquad \forall \; j \in J_{y} \; \mathrm{y} \; \forall \; j' \in J_{y'},
\end{equation*}
%
lo que permite escoger alg\'{u}n $J$ para un punto arbitrario en $\Sigma$. De esta manera se puede identificar a $\Sigma$ con $S/J$.

Aunque la acci\'{o}n de $J$ en $\Sigma$ es trivial no necesariamente lo es en la fibra $\pi^{-1} (y)$, lo que da lugar a un mapeo $J: \pi^{-1} (y) \longrightarrow \pi^{-1} (y)$ que conmuta con la acci\'{o}n de $G$ en el haz. Y como la acci\'{o}n derecha de $G$ en las fibras del haz $P$ es transitiva, a cada punto $p \in \pi^{-1}(y)$ en la fibra es posible asignar un homomorfismo $\lambda_{p}$ que va del subgrupo de isotrop\'{i}a $J$ al grupo de estructura $G$ definido por $j \cdot p = p \cdot \lambda_{p} (j)$. Dado $p' = p \cdot g$ se tiene
%
\begin{equation*}
p' \cdot \lambda_{p'} (j) = j \cdot (p \cdot g) = (j \cdot p) \cdot g = (p \cdot \lambda_{p} (j)) \cdot g = p' \cdot \mathrm{Ad}_{g^{-1}} (\lambda_{p} (j)),
\end{equation*}
%
lo que implica que
%
\begin{equation}
\label{eq:conj}
\lambda_{p'} (j) = \mathrm{Ad}_{g^{-1}} (\lambda_{p} (j)).
\end{equation}
%
A continuaci\'{o}n se muestra la propiedad de homomorfismo:
%
\begin{equation*}
p \cdot \lambda_{p} (j_{1} j_{2}) = (j_{1} j_{2}) \cdot p = j_{1} \cdot (p \cdot \lambda_{p} (j_{2})) = (p \cdot \lambda_{p} (j_{2})) \cdot \mathrm{Ad}_{\lambda(j_{2})^{-1}} \lambda_{p} (j_{1}) = p \cdot (\lambda_{p} (j_{1}) \lambda_{p} (j_{2})).
\end{equation*}
%
De esta manera se tiene una clase de equivalencia $[\lambda]$ de homomorfismos $\lambda: J \longrightarrow G$ dada por la conjugaci\'{o}n \eqref{eq:conj}, donde cambiar $\lambda$ dentro de la clase equivale a un cambio de norma. El homomorfismo del \'{a}lgebra de Lie inducido $\mathfrak{j} \longrightarrow \mathfrak{g}$ tambi\'{e}n ser\'{a} denotado por $\lambda$.

Ahora bien, escogiendo un homomorfismo $\lambda: J \longrightarrow G$, se contruye el sub-haz principal reducido
%
\begin{equation}
\label{eq:ReduceBundleQ}
\tilde{Q}(\tilde{\Sigma}, Z_{\lambda}, \tilde{\pi}) = \{p \in P \vert_{\tilde{\Sigma}} \; \vert \; \lambda_{p} = \lambda\},
\end{equation}
%
con grupo de estructura $Z_{\lambda}$ el centralizador de $\lambda(J) < G$, esto es: $Z_{\lambda} = \{g \in G \; \vert \; gh = hg \; \forall \; h \in \lambda(J)\}.$ Y $P \vert_{\tilde{\Sigma}}$ es el haz restringido a $\tilde{\Sigma}$.

Los elementos $[\lambda]$ y $\tilde{Q}$ clasifican de manera \'{u}nica al haz $P$ \cite{Brodbeck}. Esto es importante porque el par $([\lambda], \tilde{Q})$ y el Teorema \autoref{thm:reductive} enunciado m\'{a}s adelante nos ayudar\'{a}n a reconstruir las conexiones invariantes bajo la acci\'{o}n de $S$.

Antes de continuar, es importante hacer unas observaciones. Por un lado se tiene que un elemento $X$ en $\mathfrak{s}$ induce un campo vectorial $\tilde{X}$ en $P$ mediante $\tilde{X}(f) = d(\exp(t X) \cdot f)/dt \vert_{t=0}$, donde $f$ es una funci\'{o}n diferenciable en $P$. Y por otro lado, para una conexi\'{o}n $\omega'$ en $P$ valuada en el \'{a}lgebra de Lie $\mathfrak{g}$, se tiene que $\omega'(\tilde{X}) = A$ con $A \in \mathfrak{g}$. Esto nos permite construir un mapeo $\Lambda'$ que vaya directamente de $\mathfrak{s}$ a $\mathfrak{g}$ como se muestra a continuaci\'{o}n en el diagrama:
%
\begin{center}
\begin{tikzpicture}[node distance=2.0cm,auto]
\node (A) {$X \in \mathfrak{s}$};
\node (B) [right=of A] {$\tilde{X} \in TP$};
\node (C) [below=of B] {$A \in \mathfrak{g}$};

\path[->, >=stealth] (A) edge node {$\exp$} (B);
\path[->, >=stealth] (A) edge node {$\Lambda'$} (C);
\path[->, >=stealth] (B) edge node {$\omega'$} (C);
\end{tikzpicture}
\end{center}

\begin{pro}
\label{pro:LinMapAndConnCorrespondence}
Sea el grupo de Lie $S < \mathrm{Aut} (P)$ transitivo, sea $J$ el subgrupo de isotrop\'{i}a de $S$ en $y = \pi(p)$ y sea $\omega$ una conexi\'{o}n en $P$ invariante bajo $S$, i.e. $s^{*}(\omega) = \omega$ para todo $s \in S$. Definimos un mapeo lineal $\Lambda: \mathfrak{s} \longrightarrow \mathfrak{g}$ por
%
\begin{equation}
\label{eq:correspondence}
\Lambda (X) = \omega (\tilde{X}),
\end{equation}
%
con $X \in \mathfrak{s}$ y $\tilde{X}$ el campo vectorial en $P$ inducido por $X$. Entonces,
%
\begin{align}
\label{eq:LambdaCond1}
& \Lambda (X) = \lambda (X) & \mathrm{para} \; X \in \mathfrak{j}; \\
\label{eq:LambdaCond2}
& \Lambda (\mathrm{ad}_{j} (X)) = \mathrm{ad}_{\lambda (j)} (\Lambda (X)) \qquad & \mathrm{para} \; j \in J \; \mathrm{y} X \in \mathfrak{s},
\end{align}
%
donde $\mathrm{ad}_{j}$ es la representaci\'{o}n adjunta de $J$ en $\mathfrak{s}$ y $\mathrm{ad}_{\lambda (j)}$ la representaci\'{o}n adjunta de $G$ en $\mathfrak{g}$.
\end{pro}

%\begin{thm}
%\label{thm:theorem1}
%Si el subgrupo de Lie $S < \mathrm{Aut} (P)$ es transitivo y si $J$ es el grupo de isotrop\'{i}a de $S$ en $y = \pi(p)$, entonces existe una correspondencia 1:1 entre el conjunto de conexiones invariantes $\omega$ en $P$ bajo la acci\'{o}n de $S$ y el conjunto de mapeos lineales $\Lambda: \mathfrak{s} \longrightarrow \mathfrak{g}$, el cual satisface las condiciones \eqref{eq:LambdaCond1} y  \eqref{eq:LambdaCond2}. La correspondecia est\'{a} dada por \eqref{eq:correspondence}.
%\end{thm}

\begin{thm}
\label{thm:reductive}
Si $\mathfrak{s}$, en la Proposici\'{o}n \autoref{pro:LinMapAndConnCorrespondence}, admite un subespacio $\mathfrak{j}_{\perp}$ tal que $\mathfrak{s} = \mathfrak{j} \oplus \mathfrak{j}_{\perp}$ y $\mathrm{ad}_{J} (\mathfrak{j}_{\perp}) = \mathfrak{j}_{\perp}$. Entonces existe una correspondecia 1:1 entre el conjunto de conexiones invariantes $\omega$ en $P$ bajo la acci\'{o}n de $S$ y el conjunto de mapeos lineales $\Lambda_{\mathfrak{j}_{\perp}}: \mathfrak{j}_{\perp} \longrightarrow \mathfrak{g}$ tal que
%
\begin{equation}
\label{eq:HiggsField}
\Lambda_{\mathfrak{j}_{\perp}} (\mathrm{ad}_{j} (X)) = \mathrm{ad}_{\lambda (j)} (\Lambda_{\mathfrak{j}_{\perp}} (X)) \qquad \mathrm{para} \; j \in J \; \mathrm{y} X \in \mathfrak{j}_{\perp},
\end{equation}
%
con la correspondencia dada por \eqref{eq:correspondence} via
%
$$
\Lambda (X) = 
\begin{cases}
\lambda (X) \qquad & \mathrm{si} \; X\in \mathfrak{j}; \\
\Lambda_{\mathfrak{j}_{\perp}} (X) \qquad & \mathrm{si} \; X \in \mathfrak{j}_{\perp}.
\end{cases}
$$
%
\end{thm}

Ver la demostraci\'{o}n de la Proposici\'{o}n \autoref{pro:LinMapAndConnCorrespondence} y del Teorema \autoref{thm:reductive} en \cite{Kobayashi}.

Para los modelos homogeneos e isotr\'{o}picos que estudiaremos m\'{a}s adelante necesitamos asumir que el grupo $cociente $S/J$$S$ es semi-simple, i.e. que el \'{a}lgebra de Lie de $S$ se puede descomponer como la suma directa $\mathfrak{s} = \mathfrak{j} \oplus \mathfrak{j}_{\perp}$ con $\mathrm{ad}_{J}(\mathfrak{j_{\perp}}) = \mathfrak{j_{\perp}}$.

De la descomposici\'{o}n $\Sigma = \tilde{\Sigma} \times S/J$ se tiene
%
\begin{equation}
\omega = \tilde{\omega} + \omega_{S/J},
\end{equation}
%
donde $\tilde{\omega}$ es una conexi\'{o}n en el sub-haz principal reducido $\tilde{Q}(\tilde{\Sigma}, Z_{\lambda}, \tilde{\pi})$ y $\omega_{S/J}$ es una uno-forma invariante en $S/J$. Sin embargo, en este trabajo $\tilde{\omega}$ no nos concierne pues al fijar un grupo de isotrop\'{i}a $J$ (sujeto a conjugaci\'{o}n) y una clase de equivalencia $\lambda$ tambi\'{e}n fijamos un punto $y \in \Sigma$ entonces el espacio base est\'{a} representado por ese punto, $\tilde{\Sigma} \cong \{y\}$, por lo que en realidad no existe una definici\'{o}n de $\tilde{\omega}$ as\'{i} que toda la informaci\'{o}n de $\omega$ est\'{a} contenida en $\omega_{S/J}$.

Utilizando la forma de Maurer-Cartan $\theta_{(\mathrm{MC})}$ en $S$ y escogiendo un encaje $\iota: S/J \hookrightarrow S$, la forma de una conexi\'{o}n invariante, dada la clasificaci\'{o}n por $[\lambda]$ del haz principal $P$ sim\'{e}trico bajo la acci\'{o}n de $S$, se sigue de la relaci\'{o}n \eqref{eq:correspondence}:
%
\begin{equation*}
\omega_{S/J} (\tilde{X}) = \Lambda (X) = \Lambda (\iota^{*} \theta_{\mathrm{MC}} (\tilde{X})) = \Lambda \circ \iota^{*} \theta_{\mathrm{MC}} (\tilde{X})
\end{equation*}
%
que $\omega_{S/J}$ se puede escribir como:
%
\begin{equation}
\omega_{S/J} = \Lambda \circ \iota^{*} \theta_{\mathrm{MC}}.
\end{equation}