\section{Antecedentes}

\subsection{Clasificaci�n de Haces Principales sim�tricos}

Sea un haz principal $\pi: P \longrightarrow \mathcal{M}$ con grupo compacto de estructura $G$ y sea $S < \mathrm{Aut}(P)$ un grupo compacto de simetr�a actuando en el haz principal $P$, el cual, a trav�s del mapeo de proyecci�n $\pi$, induce una S-acci�n en la base $\mathcal{M}$. Dado que el grupo de simetr�a es compacto existe una subvariedad densa y abierta $\mathcal{M}_{(J)} \subset \mathcal{M}$, que localmente est� foliada por las �rbitas de $S$. Esto es, $\mathcal{M}_{(J)} \cong  \mathcal{M}_{(J)}/S \times S/J$ para un subgrupo compacto $J \subset S$. El espacio base $\mathcal{M}$ se puede descomponer de la siguiente forma, $$\mathcal{M} = \mathcal{\tilde{M}} \times S/J,$$ donde $\mathcal{\tilde{M}}$ es una variedad conexa, y la acci�n de $S$ en $\mathcal{M}$ es tal que las �rbitas est�n dadas por $S(m) \cong S/J$ para todo $m \in \mathcal{M}$.

\begin{thm}
\label{thm:theorem1}
A partir de las suposiciones anteriores, el haz principal $\pi: P \longrightarrow \mathcal{M}$, con grupo de estructura $G$, sim�trico bajo la acci�n del grupo $S$, est� clasificado por un homomorfismo $\lambda: J \longrightarrow G$ y un haz principal $\tilde{\pi}: \tilde{Q} \longrightarrow \mathcal{\tilde{M}}$ con grupo de estructura $Z$, donde $Z$ es el centralizador del subgrupo $\lambda (J) \subset G$.
\end{thm}

El haz principal $\tilde{Q}$ es subhaz de $P$ tal que $$\tilde{Q} = \{p \in P|_{\mathcal{\tilde{M}}} : \lambda_{p} = \lambda\},$$ donde el homomorfismo $\lambda_{p}$ es un mapeo asignado a cada punto $p \in \pi^{-1} (m)$ que va del subgrupo de isotrop�a $J$ al grupo de estructura $G$ definido por $p \cdot j = p \cdot \lambda_{p} (j)$. Gracias a la conmutatividad de la acci�n de $J$ con la acci�n derecha de $G$ en $P$, el homomorfismo $\lambda_{p}: J \longrightarrow G$ obedece la relaci�n $\lambda_{p'} = \mathrm{Ad}_{g^{-1}} \circ \lambda_{p}$ para alg�n punto $p' = p \cdot g$ en la misma fibra, $$p' \cdot \lambda_{p'}(j) = (p \cdot g) \cdot j = (p \cdot j) \cdot g = (p \cdot \lambda_{p} (j)) \cdot g = p' \cdot \mathrm{Ad}_{g^{-1}} (\lambda_{p} (j)).$$

\subsection{Clasificaci�n de Conexiones invariantes}

Sea $\omega$ una conexi�n invariante en el haz principal $P$, clasificado por el par $(\lambda, \tilde{Q})$, bajo la acci�n de $S$, i.e. $s^{*}(\omega) = \omega$ para todo $s \in S$. De la restricci�n, la conexi�n $\omega$ induce una conexi�n $\tilde{\omega}$ en el subhaz $\tilde{Q}$ con valores en el �lgebra de Lie del grupo de estructura $Z$. Esto es,
%
\begin{align*}
j^{*} (\omega_{p}(v)) = & \omega_{p \cdot j} (j_{*} (v))
= \omega_{p \cdot \lambda_{p}(j)}  (\lambda_{p}(j)_{*} (v)) \\
= & \lambda_{p}(j)^{*}  (\omega_{p}(v))
= \mathrm{Ad}_{\lambda (j)^{-1} *} (\omega_{p}(v)),
\end{align*}
%
pero $\omega$ es invariante bajo la acci�n de $S$ y $J \subset S$, entonces $\mathrm{Ad}_{\lambda(j)^{-1} *}(\omega_{p} (v)) = \omega_{p} (v)$ para todo $j \in J$ y para un vector $v \in T_{p}P$, tal que $\pi_{*} (v) \in \varrho_{*} (T_{\pi(p)} \mathcal{\tilde{M}})$ donde $\varrho$ es el encaje de $\mathcal{\tilde{M}}$ en $\mathcal{M}$. Lo que muestra efectivamente que $\omega_{p}(v) \in T_{e}Z$ y $\omega$ puede ser restringida a una conexi�n en el haz $\tilde{Q}$ con grupo de estructura $Z$ \cite{Bojowald2005}.

\begin{thm}%[Teorema Generalizado de Wang]
Sea $\pi: P \longrightarrow \mathcal{M}$ un haz principal $S$-sim�trico con grupo de estructura $G$ clasificado por el par $(\lambda, \tilde{Q})$ (teorema \ref{thm:theorem1}) y sea $\omega$ una conexi�n invariante en $P$ bajo la acci�n de $S$. Entonces $\omega$ est� clasificada de manera �nica por una conexi�n $\tilde{\omega}$ en $\tilde{Q}$ y un mapeo lineal $\tilde{\phi}$ que va de $T_{e}J_{\perp}$, el complemento de $T_{e}J \subset T_{e}S$, a $T_{e}G$. El mapeo lineal $\phi$ est� restringido a la relaci�n
%
\begin{equation}
\label{eq:HiggsField}
\mathrm{Ad}_{\lambda(j)} \circ \tilde{\phi} (X) = \tilde{\phi} \circ Ad_{j} (X),
\end{equation}
%
para todo $j \in J$ y $j \in T_{e}J_{\perp}$.
\end{thm}

Con lo anterior, la conexi�n $\omega$ puede ser reconstruida a partir de la clasificaci�n dada por el par $(\tilde{\omega}, \tilde{\phi})$ . De la descomposici�n $\mathcal{M} = \mathcal{\tilde{M}} \times S/J$ se tiene
%
\begin{equation}
\omega = \tilde{\omega} + \omega_{S/J},
\end{equation}
%
donde $\omega_{S/J}$, usando la forma de Maurer-Cartan $\theta_{(\mathrm{MC})}$ en $S$ y el encaje $\iota: S/J \hookrightarrow S$, se puede escribir como
%
\begin{equation}
\omega_{S/J} = \phi \circ \iota^{*} \theta_{\mathrm{MC}}.
\end{equation}