\section{Antecedentes}

\subsection{Clasificaci�n de Haces Principales sim�tricos}

Sea un haz principal $\pi: P \longrightarrow \mathcal{M}$ con grupo compacto de estructura $G$ y sea $S < \mathrm{Aut}(P)$ un grupo compacto de simetr�a actuando en el haz principal $P$, el cual, a trav�s del mapeo de proyecci�n $\pi$, induce una S-acci�n en la base $\mathcal{M}$. Dado que el grupo de simetr�a es compacto existe una subvariedad densa y abierta $\mathcal{M}_{(K)} \subset \mathcal{M}$, que localmente est� foliada por las �rbitas de $S$. Esto es, $\mathcal{M}_{(K)} \cong  \mathcal{M}_{(K)}/S \times S/K$ para un subgrupo compacto $K \subset S$. El espacio base $\mathcal{M}$ se puede descomponer de la siguiente forma, $$\mathcal{M} = \mathcal{\tilde{M}} \times S/K,$$ donde $\mathcal{\tilde{M}}$ es una variedad conexa, y la acci�n de $S$ en $\mathcal{M}$ es tal que las �rbitas est�n dadas por $S(m) \cong S/K$ para todo $m \in \mathcal{M}$.

\begin{thm}
\label{thm:theorem1}
A partir de las suposiciones anteriores, el haz principal $\pi: P \longrightarrow \mathcal{M}$, con grupo de estructura $G$, sim�trico bajo la acci�n del grupo $S$, est� clasificado por un homomorfismo $\lambda: K \longrightarrow G$ y un haz principal $\tilde{\pi}: \tilde{Q} \longrightarrow \mathcal{\tilde{M}}$ con grupo de estructura $Z$, donde $Z$ es el centralizador del subgrupo $\lambda (K) \subset G$.
\end{thm}

El haz principal $\tilde{Q}$ es subhaz de $P$ tal que $$\tilde{Q} = \{p \in P|_{\mathcal{\tilde{M}}} : \lambda_{p} = \lambda\},$$ donde el homomorfismo $\lambda_{p}$ es un mapeo asignado a cada punto $p \in \pi^{-1} (m)$ que va del subgrupo de isotrop�a $K$ al grupo de estructura $G$ definido por $k \cdot p = p \cdot \lambda_{p} (k)$. El homomorfismo $\lambda_{p}: K \longrightarrow G$ obedece la relaci�n $\lambda_{p'} = \mathrm{Ad}_{g^{-1}} \circ \lambda_{p}$ para alg�n punto $p' = p \cdot g$ en la misma fibra, $$p' \cdot \lambda_{p'}(k) = k \cdot (p \cdot g) = (k \cdot p) \cdot g = (p \cdot \lambda_{p} (k)) \cdot g = p' \cdot \mathrm{Ad}_{g^{-1}} (\lambda_{p} (k)).$$

\subsection{Clasificaci�n de Conexiones invariantes}

Sea $\omega$ una conexi�n invariante en el haz principal $P$, clasificado por el par $(\lambda, \tilde{Q})$, bajo la acci�n de $S$, i.e. $s^{*}(\omega) = \omega$ para todo $s \in S$. De la restricci�n, la conexi�n $\omega$ induce una conexi�n $\tilde{\omega}$ en el subhaz $\tilde{Q}$ con valores en el �lgebra de Lie del grupo de estructura $Z$. Esto es,
%
\begin{align*}
k^{*} (\omega_{p}(v)) = & \omega_{k \cdot p} (k_{*} (v))
= \omega_{p \cdot \lambda_{p}(k)}  (\lambda_{p}(k)_{*} (v)) \\
= & \lambda_{p}(k)^{*}  (\omega_{p}(v))
= \mathrm{Ad}_{\lambda (k)^{-1} *} (\omega_{p}(v)),
\end{align*}
%
para todo $k \in K$ y para un vector $v \in T_{p}P$

\begin{thm}[Teorema Generalizado de Wang]
Sea $\pi: P \longrightarrow \mathcal{M}$ un haz principal $S$-sim�trico con grupo de estructura $G$ clasificado por el par $(\lambda, \tilde{Q})$ (teorema \ref{thm:theorem1}) y sea $\omega$ una conexi�n invariante en $P$ bajo la acci�n de $S$. Entonces $\omega$ est� clasificada por una conexi�n $\tilde{\omega}$ en $\tilde{Q}$ y un campo escalar $\tilde{\phi}$ sobre $\tilde{Q}$ con valores en el subespacio lineal de $T_{e}G \otimes T_{e}K^{*}_{\perp}$ definido por $$\mathrm{Ad}_{\lambda(k)} \circ \tilde{\phi} = \tilde{\phi} \circ Ad_{k},$$ para todo $k \in K$. $T_{e}K_{\perp}$ es el complemento ortogonal de $T_{e}K \subset T_{e}S$ con respecto a la m�trica de Cartan-Killing.
\end{thm}