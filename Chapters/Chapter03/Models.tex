\section{Modelos Homog\'{e}neos}

Los modelos de Bianchi clase A describen todos los posibles espacios homogeneos \cite{Wald}, en estos modelos el grupo de simetr\'{i}a $S$ act\'{u}a libre y transitivamente en el espacio base $\Sigma$, por lo que $\Sigma$ puede ser identificado con $S$ de manera local. Por otro lado, tenemos que una conexi\'{o}n invariante $A$ es una uno-forma que toma valores en el \'{a}lgebra de Lie $\mathfrak{g}$ del grupo de estructura $G$, el cual para las Variables de Conexi\'{o}n es $G=\mathrm{SU}(2)$.

Denotamos a los tres generadores de $\mathfrak{s}$ como $T_{\hat{I}}$, $\hat{I}=1,2,3$, tal que $\left[ T_{\hat{I}}, T_{\hat{J}} \right] = C^{\hat{K}}_{\hat{I}\hat{J}} T_{\hat{K}}$, donde $C^{\hat{K}}_{\hat{I}\hat{J}}$ son las constantes de estructura de $\mathfrak{s}$ que satisfacen $C^{\hat{K}}_{\hat{I}\hat{K}} = 0$ (para modelos clase A por definici\'{o}n). Por otro lado, ya existe un sentido general de invariancia con la forma de Maurer-Cartan en $S$ dada por $\theta_{\mathrm{MC}} = \omega^{\hat{I}} \otimes T_{\hat{I}}$ con $\omega^{\hat{I}} = \omega^{\hat{I}}_{a} dx^{a}$ uno-formas invariantes por la izquierda en $S$ que cumplen las ecuaciones de Maurer-Cartan
%
\begin{equation}
d \omega^{\hat{I}} = -\frac{1}{2} C^{\hat{I}}_{\hat{J} \hat{K}} \omega^{\hat{I}} \wedge \omega^{\hat{J}}.
\end{equation}
%
As\'{i} que cada $\omega^{\hat{I}}$ obtenida de esta manera es una uno-forma en $S$ tal que $s^{*}(\omega^{\hat{I}}) = \omega^{\hat{I}}$ para todo $s$ en $S$.

Ahora bien, tenemos que $\theta_{\mathrm{MC}}$ toma valores en el \'{a}lgebra de Lie $\mathfrak{s}$. Sin embargo, la conexi\'{o}n invariante $A$ debe tomar valores en el \'{a}lgebra de Lie del grupo de estructura $G$, es decir, en $\mathfrak{g}$ y obedecer la misma condici\'{o}n de invariancia que $\theta_{\mathrm{MC}}$. Aplicando el mapeo lineal $\Lambda: \mathfrak{s} \longrightarrow \mathfrak{g}$ podemos obtener la conexi\'{o}n invariante $A$ en $\mathfrak{g}$.

Por la acci\'{o}n libre el grupo de isotrop\'{i}a es $J=\{1\}$, lo que implica que todos los homomorfismos $\lambda: J \rightarrow G$ est\'{a}n dados por $1^{(J)} \longmapsto 1^{(G)}$, y es conveniente usar el encaje $\iota = \mathrm{id}: S/J \hookrightarrow S$. De este modo, la conexi\'{o}n invariante toma la forma
%
\begin{equation}
A = \Lambda \circ \theta_{\mathrm{MC}} = \Lambda^{I}_{\hat{J}} \tau_{I} \omega^{\hat{J}} =  \Lambda^{I}_{\hat{J}} \omega^{\hat{J}}_{a} \tau_{I} dx^{a}=A^{I}_{a} \tau_{I} dx^{a},
\end{equation}
%
con $\tau_{I} = -i \sigma_{I}/2$ generando $\mathfrak{su}(2)$ ($\sigma_{I}$ son las matrices de Pauli).

Para $J=\{1\}$ la condici\'{o}n \eqref{eq:HiggsField} est\'{a} vac\'{i}a, entonces el mapeo lineal est\'{a} dado por $\Lambda: \mathfrak{j}_{\perp} = \mathfrak{s} \longrightarrow \mathfrak{su}(2)$, $T_{\hat{I}} \longmapsto \Lambda(T_{\hat{I}}) =: \Lambda^{J}_{\hat{I}} \tau_{J}$. De esta manera tenemos que las componentes de $A$ tienen la forma
%
\begin{equation}
A^{I}_{a} = \Lambda^{I}_{\hat{K}} \omega^{\hat{K}}_{a},
\end{equation}
%
y las componentes de su momento can\'{o}nicamente conjugado, i.e. la triada densitizada $E$, son
%
\begin{equation}
E^{a}_{I} = \sqrt{h_{0}} \tilde{p}^{\hat{K}}_{I} X^{a}_{\hat{K}},
\end{equation}
%
con $\tilde{p}^{\hat{K}}_{I}$ can\'{o}nicamente conjugado a $\Lambda^{I}_{\hat{K}}$, $h_{0} = \det(\omega^{\hat{K}}_{a})^{2}$ es el determinante de la m\'{e}trica invariante por la izquierda\footnotemark que densitiza a $E^{a}_{I}$, y $X_{\hat{I}}$ campos vectoriales invariantes por la izquierda tal que $\omega^{\hat{I}}(X_{\hat{J}}) = \delta^{\hat{I}}_{\hat{J}}$ y con $[X_{\hat{I}}, X_{\hat{J}}] = C^{\hat{K}}_{\hat{I} \hat{J}} X_{\hat{K}}$.
\footnotetext{$(h_{0})_{ab} := \sum\limits_{\hat{K}} \omega^{\hat{K}}_{a} \omega^{\hat{K}}_{b}.$}

La estructura simpl\'{e}ctica se calcula de
%
\begin{equation}
\frac{1}{8 \pi \beta G_{0}} \int\limits_{\Sigma} d^{3} x \dot{A}^{I}_{a} E^{a}_{I} = \frac{1}{8 \pi \beta G_{0}} \int\limits_{\Sigma} d^{3} x \sqrt{h_{0}} \dot{\Lambda}^{I}_{\hat{K}} \tilde{p}^{\hat{J}}_{I} \omega^{\hat{K}} (X_{\hat{J}}) = \frac{V_{0}}{8 \pi \beta G_{0}} \dot{\Lambda}^{I}_{\hat{J}} \tilde{p}^{\hat{J}}_{I},
\end{equation}
%
y se obtiene
%
\begin{equation}
\{\Lambda^{I}_{\hat{K}}, \tilde{p}^{\hat{J}}_{L} \} = 8 \pi \beta G_{0} V_{0} \delta^{I}_{L} \delta^{\hat{J}}_{\hat{K}},
\end{equation}
%
con $G_{0}$ como la constante gravitacional, $\beta$ como el par\'{a}metro de Immirzi y con volumen $V_{0} := \int\limits_{\Sigma} d^{3} x  \sqrt{h_{0}}$. Es conveniente definir $\phi^{I}_{\hat{J}} := V_{0}^{1/3} \Lambda^{I}_{\hat{J}}$ y $p^{\hat{I}}_{J} := V_{0}^{2/3} \tilde{p}^{\hat{I}}_{J}$ para hacer la estructura simpl\'{e}ctica independiente de $V_{0}$ y as\'{i} respetar la independencia de fondo.

\section{Modelos Isotr\'{o}picos}

Al imponer m\'{a}s condiciones de simetr\'{i}a se introducen subgrupos de isotrop\'{i}a no triviales, $J \cong \mathrm{U}(1)$ para modelos isotr\'{o}picos respecto a un eje, llamados sim\'{e}tricos rotacionalmente de manera local (LRS por sus siglas en ingl\'{e}s), y $J \cong \mathrm{SU}(2)$ para modelos de Bianchi tipo I y IX restringidos a m\'{e}tricas isotr\'{o}picas, es decir, que son totalmente isotr\'{o}picos. Escribiendo al grupo de simetr\'{i}a como un producto semi-directo $S = J \rtimes_{\rho} N$, con el grupo de isotro\'{i}a $J$ y el subgrupo de traslaciones $N$, se puede utilizar la t\'{e}cnica ya discutida para encontrar la forma de la conexi\'{o}n invariante. La composici\'{o}n en este grupo est\'{a} definida como $(n_{1}, j_{1})(n_{2}, j_{2}) := (n_{1} \rho(j_{1}) (n_{2}), j_{1} j_{2} )$ que depende del homomorfismo\footnotemark $\rho: J \longrightarrow \mathrm{Aut}(N)$, mientras que el inverso est\'{a} dado por $(n, j)^{-1} = (\rho(j^{-1}) n^{-1}, j^{-1})$.
\footnotetext{Esto es $\rho(j): N \longrightarrow N$ es un isomorfismo para todo $j \in J$.}

Ahora bien, para determinar la forma de conexiones invariantes vemos que la forma de Maurer-Cartan en $S$ est\'{a} dada por
%
\begin{align*}
\label{eq:MC-S}
\theta^{(S)}_{\mathrm{MC}} (n, j) = & (n, j)^{-1} d(n, j) = (\rho(j^{-1}) n^{-1}, j^{-1}) (dn, dj) \\
= & (\rho(j^{-1}) n^{-1} \rho(j^{-1}) dn, j^{-1} dj) = (\rho(j^{-1}) (n^{-1} dn), j^{-1} dj) \\
= & \left(\rho(j^{-1}) \theta^{(N)}_{\mathrm{MC}} (n), \theta^{(J)}_{\mathrm{MC}} (j) \right),
\end{align*}
%
aqu\'{i} $\theta^{(N)}_{\mathrm{MC}}$ y $\theta^{(J)}_{\mathrm{MC}}$ denotan las forma de Maurer-Cartan en $N$ y $J$ respectivamente. Si el encaje $\iota: S/J = N \hookrightarrow S$ se escoge como $\iota(n) = (n, 1)$, tenemos: $\iota^{*} \theta^{(S)}_{\mathrm{MC}} = \theta^{(N)}_{\mathrm{MC}}$. Usando los generadores $T_{\hat{I}}$ de $\mathfrak{n} = \mathfrak{j}_{\perp}$ y uno-formas invarinates $\omega^{\hat{I}}$ en $N$, se puede escribir $\theta^{(N)}_{\mathrm{MC}} = \omega^{\hat{I}} \otimes T_{\hat{I}}$, con $\omega^{\hat{I}} = \omega^{\hat{I}}_{a} dx^{a}$ uno-formas invariantes por la izquierda en $N$. De esta manera, la conexi\'{o}n toma la forma
%
\begin{equation}
\label{eq:wN}
A = \Lambda \circ \iota^{*} \theta^{(S)}_{\mathrm{MC}} = \Lambda^{I}_{\hat{J}} \omega^{\hat{J}} \tau_{I} =  \Lambda^{I}_{\hat{J}} \omega^{\hat{J}}_{a} \tau_{I} dx^{a} = A^{I}_{a} \tau_{I} dx^{a},
\end{equation}
%
donde las componentes $\Lambda^{I}_{\hat{J}}$ est\'{a}n definidas por $\Lambda(T_{\hat{J}}) = \Lambda^{I}_{\hat{J}} \tau_{I}$; $\tau_{I}$ son los generadores de $\mathfrak{su} (2)$ que est\'{a}n dados en t\'{e}rminos de las matrices da Pauli: $\tau_{I} = -i \sigma_{I}/2$ (y adem\'{a}s forman una base en $\mathfrak{s}$). N\'{o}tese que \eqref{eq:wN} tiene la misma forma que el modelo anterior, pero en este caso la condici\'{o}n \eqref{eq:HiggsField} no est\'{a} vac\'{i}a, por lo que se obtiene un subconjunto de conexiones isotr\'{o}picas. Para encontrar la soluci\'{o}n a est\'{a} condici\'{o}n se debe trata al modelo LRS y al modelo totalmente isotr\'{o}pico por separado. En el modelo LRS se toma como sistema generador a $\mathfrak{j} = \langle \tau_{3} \rangle$ mientras que en el modelo totalmente isotr\'{o}pico, el sistema generador es $\mathfrak{j} = \langle \tau_{1}, \tau_{2}, \tau_{3} \rangle$. La ecuaci\'{o}n \eqref{eq:HiggsField} se puede escribir de forma infinitesimal como
%
\begin{equation*}
\Lambda (\mathrm{ad}_{\tau_{\hat{I}}} (T_{\hat{J}})) = \mathrm{ad}_{d \lambda (\tau_{\hat{I}})} \Lambda (T_{\hat{J}}) = [d \lambda (\tau_{\hat{I}}), \Lambda (T_{\hat{J}})]
\end{equation*}
%
($\hat{I} = 3$ para los modelos LRS, $\hat{I} = 1,2,3$ para los modelos totalmente isotr\'{o}picos). El grupo de isotrop\'{i}a $J$ rota los generadores $T_{\hat{I}}$, entonces $\mathrm{ad}_{\tau_{I}} (T_{\hat{J}}) = \epsilon_{I \hat{J} \hat{K}} T_{\hat{K}}$.

De los posibles homomorfismos $\lambda: J \longrightarrow G$, se tiene que para los modelos LRS
%
\begin{align*}
\lambda_{k}: & \mathrm{U}(1) \longrightarrow \mathrm{SU}(2) \\
& \exp(t \tau_{3}) \longmapsto \exp(k t \tau_{3}),
\end{align*}
%
para $k \in \mathbb{N}$. Entonces \eqref{eq:HiggsField} toma la forma $\epsilon_{3 \hat{I} \hat{K}} \tilde{\phi}^{J}_{\hat{K}} = k \epsilon_{3LJ} \Lambda^{L}_{\hat{I}}$, la cual s\'{o}lo tiene soluci\'{o}n no trivial para $k = 1$, que lleva a que las componentes de $\Lambda$ se puedan escribir como
%
\begin{equation}
\Lambda_{1} = \frac{1}{\sqrt{2}} \left(\tilde{a} \tau_{1} + \tilde{b} \tau_{2} \right), \qquad
\Lambda_{2} = \frac{1}{\sqrt{2}} \left(-\tilde{b} \tau_{1} + \tilde{a} \tau_{2} \right), \qquad
\Lambda_{3} = \tilde{c} \tau_{3},
\end{equation}
%
con $\tilde{a}, \, \tilde{b}, \, \tilde{c}$ n\'{u}meros arbitrarios. Las componentes del momento conjugado tienen la forma
%
\begin{equation}
\tilde{p}^{1} = \frac{1}{\sqrt{2}} \left(\tilde{p}_{a} \tau_{1} + \tilde{p}_{b} \tau_{2} \right), \qquad
\tilde{p}^{2} = \frac{1}{\sqrt{2}} \left(-\tilde{p}_{b} \tau_{1} + \tilde{p}_{a} \tau_{2} \right), \qquad
\tilde{p}^{3} = \tilde{p}_{c} \tau_{3},
\end{equation}
%
(los factores $1/\sqrt{2}$ son constantes de normalizaci\'{o}n). Y la estructura simpl\'{e}ctica est\'{a} dada por
%
\begin{equation}
\{\tilde{a}, \tilde{p}_{a}\} = \{\tilde{b}, \tilde{p}_{b}\} = \{\tilde{c}, \tilde{p}_{c}\} = \frac{8 \pi}{3} \beta G_{0} V_{0},
\end{equation}
%
y cero en cualquiera otro caso.

Para los modelos totalmente isotr\'{o}picos hay dos posibles homomorfismos $\lambda: \mathrm{SU}(2) \longrightarrow \mathrm{SU}(2)$, se tienen a $\lambda_{0}$ mapeando cualquier elemento del grupo de isotrop\'{i}a $J$ a la identidad en $G$, i.e. $j \longmapsto 1^{(G)}$, y a $\lambda_{1} = \mathrm{id}$. Para el caso $\lambda_{0}$, la ecuaci\'{o}n \eqref{eq:HiggsField} toma la forma $\epsilon_{I \hat{I} \hat{K}} \Lambda^{J}_{K} = 0$ y para $\lambda_{1}$ toma la forma $\epsilon_{I \hat{I} \hat{K}} \Lambda^{J}_{\hat{K}} = \epsilon_{ILJ} \Lambda^{L}_{\hat{I}}$, cuya soluci\'{o}n es
%
\begin{equation}
\Lambda^{J}_{\hat{I}} = \tilde{c} \delta^{J}_{\hat{I}},
\end{equation}
%
con $\tilde{c}$ un n\'{u}mero arbitrario. El momento can\'{o}nicamente conjugado se puede escribir como
%
\begin{equation}
\tilde{p}^{\hat{I}}_{J} = \tilde{p} \delta^{\hat{I}}_{J}.
\end{equation}

Por lo tanto, la conexi\'{o}n isotr\'{o}pica $A$ y la triada densitizada $E$, est\'{a}n determinadas por las componentes
%
\begin{equation}
A^{I}_{a} = \tilde{c} \delta^{I}_{a} \quad \mathrm{y} \quad E^{a}_{I} = \tilde{p} \delta^{a}_{I},
\end{equation}
%
respectivamente. Tal que la estructura simpl\'{e}tica es
%
\begin{equation}
\{A^{I}_{a}, E^{a}_{I}\} = \{\tilde{c}, \tilde{p}\} = \frac{8 \pi}{3} \beta G_{0} V_{0}.
\end{equation}
%
Nuevamente, redefiniendo $a := V_{0}^{1/3} \tilde{a}$, $b := V_{0}^{1/3} \tilde{b}$, $c := V_{0}^{1/3} \tilde{c}$ y $p_{a} := V_{0}^{2/3} \tilde{p}_{a}$, $p_{b} := V_{0}^{2/3} \tilde{p}_{b}$, $p_{c} := V_{0}^{2/3} \tilde{p}_{c}$, $p := V_{0}^{2/3} \tilde{p}$, la estructura simpl\'{e}ctica es independiente de $V_{0}$.

Como se puede observar, tanto en los modelos LRS como en los modelos totalmente isotr\'{o}picos existe una \'{u}nica soluci\'{o}n no trivial.