%---------------------------------------------------------------------------
% ECUACIONES DE CAMPO EN FORMA HAMILTONIANA
%---------------------------------------------------------------------------

\subsection{Ecuaciones del campo gravitacional}

Para obtener las ecuaciones de campo gravitacional en forma Hamiltoniana se hace la variaci\'{o}n de la acci\'{o}n\footnotemark $S = S_{G}$, con $S_{G}$ escrita en forma can\'{o}nica
\footnotetext{Por el momento estamos considerando que estamos en vac\'{i}o, i.e. $S_{M} = 0$.}
%
\begin{equation}
S_{G} = \int^{t_2}_{t_1} dt \left[ \int_{\Sigma_{t}} p^{ab} \dot{h}_{ab} \, d^{3} y - H_{G} \right],
\end{equation}
%
respecto a las variables independientes $N$, $N^{a}$, $h_{ab}$, y $p^{ab}$, sujetas a las condiciones
%
\begin{equation}
\label{eq:boundcondh}
\delta N \Big|_{S_{t}} = \delta N^{a} \Big|_{S_{t}} = \delta h_{ab} \Big|_{S_{t}} = 0,
\end{equation}
%
y sin condici\'{o}n sobre $\delta p^{ab}$ de anularse en la frontera $S_{t}$. La variaci\'{o}n da
%
\begin{equation}
\label{eq:varSG31}
\delta S_{G} = \int^{t_1}_{t_2} dt \left[ \int_{\Sigma_{t}} (p^{ab} \delta \dot{h}_{ab} + \dot{h}_{ab} \delta p^{ab}) \, d^{3} y - \delta H_{G} \right].
\end{equation}

Comenzaremos calculando la variaci\'{o}n del Hamiltoniano gravitacional $H_{G}$. Iniciamos con la variaci\'{o}n con respecto a las funciones no din\'{a}micas, $N$ y $N^{a}$:
%
\begin{equation}
\label{eq:varHcrNNa}
16 \pi \delta_{N} H_{G} = \int_{\Sigma_{t}} (- \hat{\mathcal{C}} \delta N - 2 \hat{\mathcal{C}}_{a} \delta N^{a}) \, d^{3} y,
\end{equation}
%
(por la condición \eqref{eq:boundcondh} impuesta en $\delta N \Big|_{S_{t}}$ y $\delta N^{a} \Big|_{S_{t}}$ el t\'{e}rmino de frontera se anula) con
%
\begin{equation}
\hat{\mathcal{C}} := (^{(3)}R + K^{2} - K^{ab} K_{ab}) \sqrt{h}, \qquad \hat{\mathcal{C}}_{a} := [D_{b} (K_{a}^{b} - K \delta_{a}^{b})] \sqrt{h}.
\end{equation}

Para hacer la variaci\'{o}n de $H_{G}$ respecto a $p^{ab}$ y $h_{ab}$ hay que expresar el Hamiltoniano en t\'{e}rminos de $p^{ab}$ y $h_{ab}$, para ello se usa \eqref{eq:Kph} en \eqref{eq:HG}, de este modo,
%
\begin{align}
\label{eq:16piHG}
16 \pi H_{G} & = \int_{\Sigma_{t}} \left[ N h^{-1/2} (\hat{p}_{ab} \hat{p}^{ab} - \frac{1}{2} \hat{p}^{2}) - N h^{1/2} \, ^{(3)}R - 2 N_{a} h^{1/2} D_{b} (h^{-1/2} \hat{p}^{ab}) \right] \, d^{3} y
\nonumber \\
& \quad - 2 \oint_{S_{t}} \left[ N (k - k_{0}) - N_{a} h^{-1/2} \hat{p}^{ab} r_{b} \right] \sqrt{\sigma} \, d^{2} \theta.
\end{align}
%
Aqu\'{i} se defini\'{o} $\hat{p}^{ab} = 16 \pi p^{ab}$ ($\hat{p}_{ab} = 16 \pi p_{ab}$). Haciendo la variaci\'{o}n con respecto a $\hat{p}^{ab}$, se tiene
%
\begin{align}
\label{eq:varHcrp}
(16 \pi) \delta_{p} H_{G} & = \int_{\Sigma_{t}} N h^{-1/2} \delta_{p}(\hat{p}_{ab} \hat{p}^{ab} - \frac{1}{2} \hat{p}^{2}) \, d^{3} y - 2 \delta_{p} \int_{\Sigma_{t}} N_{a} h^{1/2} D_{b} (h^{-1/2} \hat{p}^{ab}) \, d^{3} y \nonumber \\
& \quad + 2 \oint_{S_{t}} N_{a} h^{-1/2} \delta \hat{p}^{ab} r_{b} \sqrt{\sigma} \, d^{2} \theta \nonumber \\
& = \int_{\Sigma_{t}} 2 \left[ N h^{-1/2} (\hat{p}_{ab} - \frac{1}{2} \hat{p} h_{ab}) + D_{(b} N_{a)} \right] \delta \hat{p}^{ab} \, d^{3} y \nonumber \\
& = \int_{\Sigma_{t}} \mathcal{H}_{ab} \delta \hat{p}^{ab} \, d^{3} y.
\end{align}

Ahora, se hace la variaci\'{o}n de $H_{G}$ con respecto a $h_{ab}$. Lo haremos por partes, primero el t\'{e}rmino de volumen\footnotemark
\footnotetext{La integral sobre $S_{t}$ viene de aplicar el teorema de Gauss al t\'{e}rmino $2 N_{a} h^{1/2} D_{b} (h^{-1/2} \hat{p}^{ab})$ en la ecuaci\'{o}n \eqref{eq:16piHG}.}
%
\begin{align}
(16 \pi) \delta_{h} H_{\Sigma} & = \int_{\Sigma_{t}} \left[ -N h^{-1} (\hat{p}^{ab} \hat{p}_{ab} - \frac{1}{2} \hat{p}^{2}) \delta_{h} h^{1/2} + N h^{-1/2} \delta_{h} (\hat{p}^{ab} \hat{p}_{ab} - \frac{1}{2} \hat{p}^{2}) \right. \nonumber \\
& \quad \left. - N \delta_{h} (h^{1/2} \, ^{(3)}R) + 2 \delta_{h} (\hat{p}^{ab} D_{b} N_{a}) \right] \, d^{3} y - 2 \delta_{h} \oint_{S_{t}} N_{a} h^{-1/2} \hat{p}^{ab} r_{b} \sqrt{\sigma} \, d^{2} \theta.
\end{align}
%
Por la condici\'{o}n de frontera \eqref{eq:boundcondh}, la variaci\'{o}n de la integral sobre $S_{t}$ se anula. Usando $^{(3)}R = R_{ab} h^{ab}$ y las relaciones $\delta h^{ab} = -h^{ac} h^{db} \delta h_{cd}$ y $\delta_{h} h^{1/2} = \frac{1}{2} h^{1/2} h^{ab} \delta h_{ab}$, es posible mostrar que
%
\begin{align*}
\delta_{h} (h^{1/2} \, ^{(3)}R) & = - h^{1/2} (R^{ab} - \frac{1}{2}\, ^{(3)}R h^{ab}) \delta h_{ab} + h^{1/2} D_{c} (h^{ab} \delta \Gamma^{c}\,_{ab} - h^{ac} \delta \Gamma^{b}\,_{ab}) \\
& = - h^{1/2} G^{ab} \delta h_{ab} + h^{1/2} D_{c} v^{c}.
\end{align*}
%
El resto de las variaciones son:
%
\begin{equation*}
\delta_{h} (\hat{p}^{ab} \hat{p}_{ab} - \frac{1}{2} \hat{p}^{2}) = 2(\hat{p}^{a}_{c} \hat{p}^{cb} - \frac{1}{2} \hat{p} \hat{p}^{ab}) \delta h_{ab},
\end{equation*}
%
\begin{equation*}
\delta_{h} D_{b} N_{a} = (D_{b} N^{c}) \delta h_{ac} + h_{ac} N^{d} \delta \Gamma^{c}\,_{bd}.
\end{equation*}
%
Con esto se sigue que
%
\begin{align}
\label{eq:vc}
(16 \pi) \delta_{h} H_{\Sigma} & = \int_{\Sigma_{t}} \left[-\frac{1}{2} N h^{-1/2} (\hat{p}^{cd} \hat{p}_{cd} - \frac{1}{2} \hat{p}^{2}) h^{ab} + 2 N h^{-1/2} (\hat{p}^{a}_{c} \hat{p}^{bc} - \frac{1}{2} \hat{p} \hat{p}^{ab}) \right. \nonumber \\
& \quad \left. + N h^{1/2} G^{ab} + 2 \hat{p}^{c (a} D_{c} N^{b)} \right] \delta h_{ab} \, d^{3} y \nonumber \\
& \quad + \int_{\Sigma_{t}} \left(-N h^{1/2} D_{c} v^{c} + 2 \hat{p}^{b}_{c} N^{d} \delta \Gamma^{c}\,_{bd} \right) \, d^{3} y.
\end{align}

Ahora bien, conviene analizar la segunda integral por separado. Primero,
%
\begin{align*}
- \int_{\Sigma_{t}} N h^{1/2} D_{c} v^{c} \, d^{3} y & = \int_{\Sigma_{t}} (\partial_{c} N) v^{c} h^{1/2} \, d^{3} y - \oint_{S_{t}} N v^{c} r_{c} \sqrt{\sigma} \, d^{2} \theta \\
& = \int_{\Sigma_{t}} (\partial_{c} N) v^{c} h^{1/2} \, d^{3} y + \oint_{S_{t}} N h^{ab} \delta (\partial_{c} h_{ab}) r^{c} \sqrt{\sigma} d^{2} \theta \\
& = - \int_{\Sigma_{t}} (h^{ab} \partial^{d} N - h^{bd} \partial^{a} N) (D_{d} \delta h_{ab}) h^{1/2} \, d^{3} y \\
& \quad + \oint_{S_{t}} N h^{ab} (\partial_{c} \delta h_{ab}) r^{c} \sqrt{\sigma} d^{2} \theta \\
& = \int_{\Sigma_{t}} (h^{ab} D_{d} D^{d} - D^{b} D^{a} N) \delta h_{ab} h^{1/2} \, d^{3} y \\
& \quad + \oint_{S_{t}} N h^{ab} (\partial_{c} \delta h_{ab}) r^{c} \sqrt{\sigma} d^{2}.
\end{align*}
%
Donde se us\'{o} la versi\'{o}n tridimensional de la ecuaci\'{o}n \eqref{eq:vnh} y la siguiente relaci\'{o}n:
%
\begin{equation}
\label{eq:varGamma3D}
\delta \Gamma^{c}_{ab} = \frac{1}{2} h^{cd} \left[D_{b} (\delta h_{da}) + D_{a} (\delta h_{db}) - D_{d} (\delta h_{ab}) \right],
\end{equation}
%
para tener: $v^{c} = 1/2 (h^{ab} h^{cd} - h^{ac} h^{bd}) [D_{b} (\delta h_{da}) + D_{a} (\delta h_{db}) - D_{d} (\delta h_{ab})]$, y al contraer con $\partial_{c} N$ (aprovechando la antisimetr\'{i}a en $a$ y $d$) obtener el resultado. Para el otro t\'{e}rmino de la segunda integral, tenemos que
%
\begin{align*}
\int_{\Sigma_{t}} 2 \hat{p}^{b}_{c} N^{d} \delta \Gamma^{c}\,_{bd} \, d^{3} y & = \int_{\Sigma_{t}} h^{1/2} \hat{p}^{ab} N^{d} D_{d} (\delta h_{ab}) h^{-1/2} \, d^{3} y \\
& = - \int_{\Sigma_{t}} D_{d} (h^{-1/2} \hat{p}^{ab} N^{d}) \delta h_{ab} h^{1/2} \, d^{3} y,
\end{align*}
%
donde tambi\'{e}n se us\'{o} \eqref{eq:varGamma3D} y depu\'{e}s, al integrar por partes, se utiliz\'{o} el teorema de Gauss y la condici\'{o}n de frontera $\delta h_{ab} \vert_{S_{t}} = 0$. As\'{i},
%
\begin{equation}
\label{eq:varHSigma}
(16 \pi) \delta_{h} H_{\Sigma} = \int_{\Sigma_{t}} \hat{\mathcal{P}}^{ab} \delta h_{ab} \, d^{3} y + \oint_{S_{t}} N h^{ab} (\partial_{c} \delta h_{ab}) r^{c} \sqrt{\sigma} d^{2}.
\end{equation}
%
Donde
%
\begin{align*}
\hat{\mathcal{P}}^{ab} & := h^{1/2} N G^{ab} - \frac{1}{2} N h^{-1/2} (\hat{p}^{cd} \hat{p}_{cd} - \frac{1}{2} \hat{p}^{2}) h^{ab} + 2 N h^{-1/2} (\hat{p}^{a}_{c} \hat{p}^{bc} - \frac{1}{2} \hat{p} \hat{p}^{ab}) \\
& \quad -h^{1/2} (D^{a} D^{b} N - h^{ab} D_{c} D^{c} N) - h^{1/2} D_{c} (h^{-1/2} \hat{p}^{ab} N^{c}) + 2 \hat{p}^{c(a} D_{c} N^{b)}.
\end{align*}

Pasando al t\'{e}rmino de frontera, al hacer la variaci\'{o}n con respecto a $h_{ab}$ se obtiene,
%
\begin{equation}
\label{eq:varHS}
(16 \pi) \delta_{h} H_{S} = -2 \oint_{S_{t}} N \delta k \sqrt{\sigma} \, d^{2} \theta = - \oint_{S_{t}} N h^{ab} (\partial_{c} \delta h_{ab}) r^{c} \sqrt{- \sigma} \, d^{2} \theta.
\end{equation}

Haciendo la suma de \eqref{eq:varHSigma} con \eqref{eq:varHS} se llega a que la variaci\'{o}n con respecto a $h_{ab}$ de $H_{G}$ es
%
\begin{equation}
\label{eq:varHcrh}
(16 \pi) \delta_{h} H_{G} = \int_{\Sigma_{t}} \hat{\mathcal{P}}^{ab} \delta h_{ab} \, d^{3} y
\end{equation}

Finalmente juntando los resultados \eqref{eq:varHcrNNa}, \eqref{eq:varHcrp} y \eqref{eq:varHcrh} se obtiene que la variaci\'{o}n del Hamiltoniano gravitacional, bajo las condiciones \eqref{eq:boundcondh}, est\'{a} dada por,
%
\begin{equation}
\label{eq:varHG}
\delta H_{G} = \int_{\Sigma_{t}} \left( \mathcal{P}^{ab} \delta h_{ab} + \mathcal{H}_{ab} \delta p^{ab} - \mathcal{C} \delta N - 2 \mathcal{C}_{a} \delta N^{a} \right) \, d^{3} y,
\end{equation}
%
con $\mathcal{P}^{ab} := \hat{\mathcal{P}}^{ab} / (16 \pi)$, $\mathcal{C} := \hat{\mathcal{C}} / (16 \pi)$ y $\mathcal{C}^{a} := \hat{\mathcal{C}}^{a} / (16 \pi)$.

Regresando a la variaci\'{o}n del funcional de acci\'{o}n gravitacional, ecuaci\'{o}n \eqref{eq:varSG31}, integrando por partes se obtiene como resultado
%
\begin{equation}
\label{varSGHamiltonianform}
\delta S_{G} = \int^{t_{2}}_{t_{1}} dt \int_{\Sigma_{t}} \left[ (\dot{h}_{ab} - \mathcal{H}_{ab}) \delta p^{ab} - (\dot{p}^{ab} + \mathcal{P}^{ab}) \delta h_{ab} + \mathcal{C} \delta N + 2 \mathcal{C}_{a} \delta N^{a} \right] \, d^{3} y.
\end{equation}
%
Imponiendo que la acci\'{o}n sea estacionaria implica
%
\begin{equation}
\label{eq:fieldequationsHamiltonianform}
\dot{h}_{ab} = \mathcal{H}_{ab}, \quad \dot{p}^{ab} = - \mathcal{P}^{ab}, \quad \mathcal{C} = 0, \quad \mathcal{C}^{a} = 0.
\end{equation}
%
Estas son las ecuaciones de Einstein (en vac\'{i}o) en la forma Hamiltoniana. Las primeras dos rigen la evoluci\'{o}n de las variables conjugadas $h_{ab}$ y $p^{ab}$; las \'{u}ltimas dos se conocen como \emph{constricciones Hamiltoniana y de difeomorfismos (o momento)} de la Relatividad General, respectivamente. Expl\'{i}citamente las constricciones son
%
\begin{equation}
\label{eq:constrictions}
\mathcal{C} = \left(^{(3)}R + K^{2} - K^{ab} K_{ab} \right) \sqrt{h} = 0, \qquad \mathcal{C}_{a} = \left[D_{b} (K^{ab} - K h^{ab}) \right] \sqrt{h} = 0.
\end{equation}
%
Entonces, en la formulaci\'{o}n Hamiltoniana para encontrar una soluci\'{o}n a las ecuaciones de Einstein se hace una foliaci\'{o}n del espacio-tiempo escogiendo la \emph{funci\'{o}n de lapso} $N$ y el \emph{vector de corrimiento} $N^{a}$, despu\'{e}s se escogen los valores iniciales para $h_{ab}$ y $K_{ab}$ de tal manera que satisfagan las constricciones (\ref{eq:constrictions}) y entonces se determina la evoluci\'{o}n de los valores iniciales usando las ecuaciones Hamiltonianas $\dot{h}_{ab} = \mathcal{H}_{ab}$ y $\dot{p}^{ab} = - \mathcal{P}^{ab}$, que equivalentemente se pueden escribir como
%
\begin{equation}
\dot{h}_{ab} = 2 N K_{ab} + \pounds_{N} h_{ab}
\end{equation}
%
y
%
\begin{equation}
\dot{K}_{ab} = D_{b} D_{a} N - N (R_{ab} + K K_{ab} - 2 K^{c}_{a} K_{bc}) + \pounds_{N} K_{ab}.
\end{equation}
