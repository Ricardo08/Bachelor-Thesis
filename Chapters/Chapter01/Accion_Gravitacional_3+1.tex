%---------------------------------------------------------------------------
% ACCIÓN GRAVITACIONAL EN EL FORMALISMO 3+1
%---------------------------------------------------------------------------

\subsection{Acci\'{o}n gravitacional en el formalismo (3+1)}

Para construir el Hamiltoniano gravitacional se realizar\'{a} la descomposici\'{o}n (3+1) del funcional de acci\'{o}n gravitacional $S_{G}$. Por ahora s\'{o}lo se considerar\'{a}n los t\'{e}rminos $I_{EH}$ e $I_{B}$, m\'{a}s adelante se reincorporar\'{a} $I_{0}$, as\'{i} que,
%
\begin{equation}
\label{eq:SGIEHIB}
S_{G} = \frac{1}{16 \pi} \left( \int_{\Omega} R \sqrt{-g} \, d^{4} x + 2 \oint_{\partial \Omega} \varepsilon K \sqrt{|h|} \, d^{3} y \right).
\end{equation}

Como en la secci\'{o}n (\ref{subsec:foliacionfrontera}), $\Omega$ tiene por frontera a $\partial \Omega$ que es la uni\'{o}n de dos hipersuperficies espacialoides $\Sigma_{t_{1}}$ y $\Sigma_{t_{2}}$, y la hipersuperficie tipo tiempo $\mathcal{B}$ (figura \ref{fig:OmegaandboundOmega}), esto es,
%
\begin{equation}
\label{eq:domegaU}
\partial \Omega  = -\Sigma_{t_{1}} \cup \Sigma_{t_{2}} \cup \mathcal{B}.
\end{equation}
%
Con lo que el t\'{e}rmino de frontera toma la forma,
%
\begin{equation}
\label{eq:IB31}
2 \oint_{\partial \Omega} \varepsilon K \sqrt{|h|} \, d^{3} y = 2 \int_{\Sigma_{t_1}} K \sqrt{|h|} \, d^{3} y - 2 \int_{\Sigma_{t_2}} K \sqrt{|h|} \, d^{3} y + 2 \int_{\mathcal{B}} \mathcal{K} \sqrt{-\gamma} \, d^{3} z.
\end{equation}

La regi\'{o}n $\Omega$ es foleada por las hipersuperficies espacialoides $\Sigma_{t}$, sobre las cuales el escalar de Ricci est\'{a} dado por
%
\begin{equation}
\label{eq:Riccionsigma}
R = ^{(3)}R + K^{ab} K_{ab} - K^{2} - 2 \nabla_{\alpha} (n^{\beta} \nabla_{\beta} n^{\alpha} - n^{\alpha} \nabla_{\beta} n^{\beta}),
\end{equation}
%
donde $^{(3)}R$ es el escalar de Ricci construido a partir de $h_{ab}$. De esta forma, el t\'{e}rmino de Einstein-Hilbert queda reescrito como
%
\begin{equation}
\int_{\Omega} R \sqrt{-g} \, d^{4} x = \int^{t_{1}}_{t_{2}} dt \left[ \int_{\Sigma_{t}} (^{(3)}R + K^{ab} K_{ab} - K^{2}) N \sqrt{h} \, d^{3} y - 2 \oint_{\partial \Omega} (n^{\beta} \nabla_{\beta} n^{\alpha} - n^{\alpha} \nabla_{\beta} n^{\beta}) \, d \Sigma_{\alpha} \right],
\end{equation}
%
donde se us\'{o} la expresi\'{o}n \eqref{eq:elemntvol31}, \eqref{eq:Riccionsigma} y el teorema de Gauss. Como la frontera es la uni\'{o}n de varias hipersuperficies, ecuaci\'{o}n \eqref{eq:domegaU}, la integral sobre $\partial \Omega$ se descompone en integrales sobre $\Sigma_{t_{1}}$, $\Sigma_{t_{2}}$ y $\mathcal{B}$. El trato de la integral sobre $\Sigma_{t_{1}}$ y $\Sigma_{t_{2}}$ es el mismo (salvo por un signo global), as\'{i} que solamente es necesario mostrar uno de estos c\'{a}lculos,
%
\begin{align}
-2 \int_{\Sigma_{t_{1}}} (n^{\beta} \nabla_{\beta} n^{\alpha} - n^{\alpha} \nabla_{\beta} n^{\beta}) \, d \Sigma_{\alpha} & = -2 \int_{\Sigma_{t_{1}}} (\nabla_{\beta} n^{\beta}) \sqrt{h} \, d^{3} y \nonumber \\
& = -2 \int_{\Sigma_{t_{1}}} K \sqrt{h} \, d^{3} y.
\end{align}
%
La integral sobre $\Sigma_{t_{2}}$ da el mismo resultado pero con signo contrario. Ambos resultados se cancelan con las integrales sobre $\Sigma_{t_{1}}$ y $\Sigma_{t_{2}}$ de \eqref{eq:IB31}. Por otro lado, la integral sobre $\mathcal{B}$ contribuye con
%
\begin{align}
-2 \int_{\mathcal{B}} (n^{\beta} \nabla_{\beta} n^{\alpha} - n^{\alpha} \nabla_{\beta} n^{\beta}) \, d \Sigma_{\alpha} & = -2 \int_{\mathcal{B}} (n^{\beta} \nabla_{\beta} n^{\alpha}) r_{\alpha} \sqrt{-\gamma} \, d^{3} z \nonumber \\
& = 2 \int_{\mathcal{B}} n^{\alpha} n^{\beta} \nabla_{\beta} r_{\alpha} \sqrt{-\gamma} \, d^{3} z.
\end{align}

Juntando los resultados, el funcional de acci\'{o}n gravitacional (a\'{u}n sin el t\'{e}rmino $I_{0}$) queda de la siguiente manera
%
\begin{equation}
\label{eq:SGIEHIB}
S_{G} = \frac{1}{16 \pi} \int^{t_{2}}_{t_{1}} dt \left[ \int_{\Sigma_{t}} (^{(3)}R + K^{ab} K_{ab} - K^{2}) N \sqrt{h} \, d^{3} y + 2 \int_{\mathcal{B}} (\mathcal{K} + n^{\alpha} n^{\beta} \nabla_{\beta} r_{\alpha}) \sqrt{-\gamma} \, d^{3} z\right].
\end{equation}

Ahora bien, a partir de las relaciones de completez \eqref{eq:Bcompltrelation} y \eqref{eq:complt} se manipula $(\mathcal{K} + n^{\alpha} n^{\beta} \nabla_{\beta} r_{\alpha})$. Primero,
%
\begin{equation}
\mathcal{K} = \gamma^{fg} \mathcal{K}_{fg} = \gamma^{fg} (e^{\alpha}_{f} e^{\beta}_{g} \nabla_{\beta} r_{\alpha}) = (g^{\alpha \beta} - r^{\alpha} r^{\beta}) \nabla_{\beta} r_{\alpha},
\end{equation}
%
que al sumar $n^{\alpha} n^{\beta} \nabla_{\beta} r_{\alpha}$, se obtiene
%
\begin{equation}
\label{eq:manipulationk}
(g^{\alpha \beta} - r^{\alpha} r^{\beta}) \nabla_{\beta} r_{\alpha} + n^{\alpha} n^{\beta} \nabla_{\beta} r_{\alpha} = \sigma^{vw} e^{\alpha}_{v} e^{\beta}_{w} (\nabla_{\beta} r_{\alpha}) = \sigma^{vw} k_{vw} = k.
\end{equation}

Sustituyendo (\ref{eq:manipulationk}) en (\ref{eq:SGIEHIB}) y adem\'{a}s, tomando en cuenta que $\mathcal{B}$ est\'{a} foleada por las superficies cerradas $S_{t}$ (as\'{i} que $\sqrt{-\gamma} d^{3} z = N \sqrt{\sigma} dt d^{2} \theta$), entonces,
%
\begin{equation}
\label{eq:LagSg}
S_{G} = \frac{1}{16 \pi} \int^{t_{2}}_{t_{1}} dt \left[ \int_{\Sigma_{t}} (^{(3)}R + K^{ab} K_{ab} - K^{2}) N \sqrt{h} \, d^{3} y + 2 \oint_{S_{t}} (k - k_{0}) N \sqrt{\sigma} \, d^{2} \theta \right].
\end{equation}

N\'{o}tese que ahora s\'{i} se incluy\'{o} $I_{0}$ al escribir el t\'{e}rmino $k_{0}$ en la integral sobre $S_{t}$. La elecci\'{o}n de $k_{0}$ como la curvatura extr\'{i}nseca de $S_{t}$ encajada en el espacio plano previene que la integral diverja en el l\'{i}mite $S_{t} \rightarrow \infty$, asegurando que $S_{G}$ est\'{e} bien definido para cualquier espacio-tiempo asint\'{o}ticamente plano.