\subsection{Espacio Fase}

La formulaci�n Hamiltoniana permite hacer un estudio del espacio fase de la Relatividad General. Habiendo hecho las descomposici�n 3+1, se tiene que las variables independientes son $N$, $N^{a}$ (�stas dos no din�micas) y $h_{ab}$ que respectivamente cada una tiene como momento conjugado a
%
\begin{align}
\Pi & = \frac{\partial \mathcal{L}_{G}}{\partial \dot{N}} = 0, \\
\Pi_{a} & = \frac{\partial \mathcal{L}_{G}}{\partial \dot{N}_{a}} = 0, \\
p^{ab} & = \frac{\partial \mathcal{L}_{G}}{\partial \dot{h}_{ab}} = \frac{\sqrt{h}}{16 \pi} (K^{ab} - K h^{ab}).
\end{align}
%
De esta manera, el espacio fase coordenado por $(N, N^{a}, h_{ab}, \Pi, \Pi_{a}, p^{ab})$, tiene la siguiente estructura simpl�ctica,
%
\begin{align}
\{N(t, y), \Pi(t, y')\} & = \delta(y - y'), \\
\{N^{a}(t, y), \Pi_{b}(t, y')\} & = \delta^{a}_{b} \delta(y - y'), \\
\{h_{ab}(t, y), p^{cd}(t, y')\} & = \delta^{c}_{(a} \delta^{d}_{b)} \delta(y - y'),
\end{align}
%
y el resto de los par�ntesis de Poisson son nulos.

Ahora bien, dado que $S_{G}$ contiene derivadas temporales de $h_{ab}$ por medio de los t�rminos que dependen de la curvatura extr�nseca, pero no hay derivadas temporales de $N$ ni de $N^{a}$, implica que el Lagrangiano sea singular y en consecuencia hay constricciones primarias. Las constricciones limitan la evoluci�n del sistema a una regi�n restringida del espacio fase, definen una superficie de constricci�n y adem�s las constricciones primarias generan transformaciones de norma sobre la superficie de constricci�n.

De la definici�n para los momentos conjugados se obtiene que las constricciones primarias son las siguientes,
%
\begin{align*}
\mathcal{C} & = \Pi = 0 \\
\mathcal{C}_{a} & = \Pi_{a} = 0. \\
\end{align*}
%
Es importante remarcar que $\Pi$ y $\Pi_{a}$ no son cero en todo el espacio fase, �nicamente en la superficie de constricci�n.

As�, la parte de la acci�n $S_{G}$ que va como $N \mathcal{C} + N^{a} \mathcal{C}_{a}$ es cero y la din�mica es pura norma. Por lo tanto, la descripci�n del sistema en el espacio fase tiene como variables fundamentales $h_{ab}$ y $p^{ab}$, mientras que la \emph{funci�n de lapso} y el \emph{vector de corrimiento} juegan el papel de multiplicadores de Lagrange.