%-------------------------------------------------------------
% LAGRANGIAN FORMULATION
%-------------------------------------------------------------

\section{Formulaci\'{o}n Lagrangiana}
\label{sec:FL}

En la formulaci\'{o}n Lagrangiana de la Relatividad General, el funcional de acci\'{o}n $S$ est\'{a} definido como la integral de una densidad lagrangiana sobre una regi\'{o}n 4-dimensional $\Omega$, de una variedad diferenciable $\mathcal{M}$ con la topolog\'{i}a del espacio-tiempo, delimitada por una hipersuperficie cerrada $\partial \Omega$.  Dicho funcional de acci\'{o}n $S$ est\'{a} formado por dos partes, la asociada al campo gravitacional $g_{\mu \nu}$, es decir la geometr\'{i}a, denotado por $S_{G}[g]$ y la parte correspondiente a los campos de materia, $S_{M}[\phi; g]$, esto es
%
\begin{equation}
\label{eq:S}
S[g; \phi] = S_{G}[g] + S_{M}[\phi; g].
\end{equation}

El campo din\'{a}mico es la m\'{e}trica del espacio-tiempo $g_{\mu \nu}$, cuyas componentes dependen de las coordenadas $x^{\alpha}$ en la regi\'{o}n $\Omega$. As\'{i}, para obtener las ecuaciones de campo gravitacional v\'{i}a el Principio de M\'{i}nima Acci\'{o}n, se introduce la variaci\'{o}n arbitraria $\delta g_{\mu \nu}(x^{\alpha})$ en $\Omega$ con la condici\'{o}n de que dicha variaci\'{o}n sea nula en la frontera $\partial \Omega$,
%
\begin{equation}
\label{eq:boundcond}
\delta g_{\mu \nu} \Big|_{\partial \Omega} = 0.
\end{equation}
Esto quiere decir que la m\'{e}trica inducida\footnotemark $h_{\alpha \beta}$ en la hipersuperficie $\partial \Omega$ se mantiene fija.
\footnotetext{La forma fundamental de la m\'{e}trica inducida en $\partial \Omega$ es $h_{\alpha \beta} = g_{\mu \nu} \frac{\partial x^{\mu}}{\partial y^{\alpha}} \frac{\partial x^{\nu}}{\partial y^{\beta}} = g_{\mu \nu} e^{\mu}_{\alpha} e^{\nu}_{\beta}$, donde $e^{\mu}_{\alpha}=\frac{\partial x^{\mu}}{\partial y^{\alpha}}$.} 

El funcional de acci\'{o}n del campo de materia, $S_{M}$, est\'{a} dada por
%
\begin{equation}
\label{eq:SM}
S_{M} [\phi; g] = \int_{\Omega} \mathcal{L}(\phi, \partial_{\alpha} \phi; g, \partial_{\alpha} g_{\mu \nu}) \sqrt{-g} \, d^{4} x.
\end{equation}
%
Por otro lado, la parte gravitacional de la acci\'{o}n $S_{G}[g]$, para facilitar los c\'{a}lculos, se divide en tres t\'{e}rminos; a saber,
%
\begin{equation}
\label{eq:SG}
S_{G}[g] = \frac{1}{16 \pi} (I_{EH}[g] + 2 I_{B}[g] + 2 I_{0}),
\end{equation}
%
el t\'{e}rmino de Einstein-Hilbert $I_{EH}[g]$, un t\'{e}rmino de frontera $I_{B}[g]$ y un t\'{e}rmino no din\'{a}mico $I_{0}$, que \'{u}nicamente influye en el valor num\'{e}rico de la acci\'{o}n mas no en las ecuaciones de movimiento, donde
%
\begin{equation*}
I_{EH}[g] = \int_{\Omega} R \sqrt{-g} \, d^{4} x, \quad I_{B}[g] = \oint_{\partial \Omega} \varepsilon K \sqrt{|h|} d^{3} y, \quad I_{0}[g] = \oint_{\partial \Omega} \varepsilon K_{0} \sqrt{|h|} d^{3} y \, .
\end{equation*}
%
Aqu\'{i}, las coordenadas $x^{\mu}$ se utilizar\'{a}n en $\Omega$ mientras que las coordenadas $y^{\alpha}$ en $\partial \Omega$. $R$ es el escalar de Ricci, $g$ es el determinante de la m\'{e}trica $g_{\mu \nu}$, $h$ es el determinante de la m\'{e}trica inducida $h_{\alpha \beta}$ y $\varepsilon \equiv n^{\mu} n_{\mu} = \pm 1$ es el m\'{o}dulo del normal unitario\footnotemark  $n_{\mu}$ a $\partial \Omega$. $K$ es la traza de la curvatura extr\'{i}nseca $K_{\alpha \beta}$ de $\partial \Omega$ y $K_{0}$ es la traza de la curvatura extr\'{i}nseca de $\partial \Omega$ encajada en el espacio-tiempo plano.
\footnotetext{El m\'{o}dulo $\varepsilon$  de $n_{\mu}$ es $+1$ donde $\partial \Omega$ es tipo-tiempo y $-1$ donde es tipo-espacio.}

Al variar la acci\'{o}n $S[g; \phi]$ con respecto a $g_{\mu \nu}$, usando el hecho de que $\delta g = g g^{\mu \nu} \delta g_{\mu \nu}$ y $g^{\mu \nu} \delta g_{\mu \nu} = -g_{\mu \nu} \delta g^{\mu \nu}$, se obtiene, primero para la parte asociada a materia, que
%
\begin{align}
\label{eq:varSML}
\delta S_{M} & = \int_{\Omega} \delta (\mathcal{L} \sqrt{-g}) \, d^{4} x \nonumber \\
& = \int_{\Omega} \left( \frac{\partial \mathcal{L}}{\partial g^{\mu \nu}} \delta g^{\mu \nu} \sqrt{-g} + \mathcal{L} \delta \sqrt{-g} \right) \, d^{4} x \nonumber \\
& = \int_{\Omega} \left( \frac{\partial \mathcal{L}}{\partial g^{\mu \nu}} - \frac{1}{2} \mathcal{L} g_{\mu \nu} \right) \delta g^{\mu \nu} \sqrt{-g} \, d^{4} x \nonumber \\
& = - \frac{1}{2} \int_{\Omega} T_{\mu \nu} \delta g^{\mu \nu} \sqrt{-g} \, d^{4} x,
\end{align}
%
definiendo el tensor de energ\'{i}a-momento como
%
\begin{equation}
\label{eq:TabL}
T_{\mu \nu} := \mathcal{L} g_{\mu \nu} - 2 \frac{\partial \mathcal{L}}{\partial g^{\mu \nu}}.
\end{equation}
%
%se puede reescribir \eqref{eq:varSML} y as\'{i} la variaci\'{o}n de la acci\'{o}n correspondiente al campo de materia queda de la siguiente manera,
%
%\begin{equation}
%\label{eq:varSMT}
%\delta S_{M} = - \frac{1}{2} \int_{\Omega} T_{\mu \nu} \delta g^{\mu \nu} \sqrt{-g} \, d^{4} x.
%\end{equation}
%
Mientras que, para la parte correspondiente a la parte gravitacional $S_{G}[g]$, comenzando con el t\'{e}rmino de Einstein-Hilbert $I_{EH}$, se tiene que
%
\begin{align}
\label{eq:IEH}
\delta I_{EH} & = \int_{\Omega} \delta(R \sqrt{-g}) \, d^{4} x \nonumber \\
%& = \int_{\Omega} \delta(g^{\mu \nu} R_{\mu \nu} \sqrt{-g}) \, d^{4} x \nonumber \\
& = \int_{\Omega} (\delta g^{\mu \nu}) R_{\mu \nu} \sqrt{-g} \, d^{4} x + \int_{\Omega} g^{\mu \nu} (\delta R_{\mu \nu}) \sqrt{-g} \, d^{4} x + \int_{\Omega} g^{\mu \nu} R_{\mu \nu} (\delta \sqrt{-g}) \, d^{4} x \nonumber \\
& = \int_{\Omega} R_{\mu \nu} (\delta g^{\mu \nu}) \sqrt{-g} \, d^{4} x + \int_{\Omega} g^{\mu \nu} (\delta R_{\mu \nu}) \sqrt{-g} \, d^{4} x -  \int_{\Omega} \frac{1}{2} R \sqrt{-g} g_{\mu \nu} \delta g^{\mu \nu} \, d^{4} x.
\end{align}

Ahora bien, considerando un marco de referencia inercial local donde los s\'{i}mbolos de Christoffel sean cero, lo que implica que $\partial_{\sigma} = \nabla_{\sigma}$, entonces,
%
\begin{equation}
\label{eq:varRicci}
\delta R_{\mu \nu} = \partial_{\lambda} \delta \Gamma^{\lambda}_{\mu \nu} - \partial_{\nu} \delta \Gamma^{\lambda}_{\mu \lambda} = \nabla_{\lambda} \delta \Gamma^{\lambda}_{\mu \nu} - \nabla_{\nu} \delta \Gamma^{\lambda}_{\mu \lambda}.
\end{equation}
%
Tomando en cuenta que $\nabla_{\sigma} g^{\mu \nu} = 0$, al multiplicar \eqref{eq:varRicci} por $g^{\mu \nu}$ se tiene que
%
\begin{equation}
\label{eq:varSRicci}
g^{\mu \nu} \delta R_{\mu \nu} = g^{\mu \nu} \nabla_{\lambda} \delta \Gamma^{\lambda}_{\mu \nu} - g^{\mu \nu}\nabla_{\nu} \delta \Gamma^{\lambda}_{\mu \lambda}  = \nabla_{\lambda}(g^{\mu \nu} \delta \Gamma^{\lambda}_{\mu \nu} - g^{\mu \lambda} \Gamma^{\nu}_{\mu \nu}),
\end{equation}
%
y definiendo
%
\begin{equation}
\label{eq:v}
v^{\lambda} := g^{\mu \nu} \delta \Gamma^{\lambda}_{\mu \nu} - g^{\mu \lambda} \delta \Gamma^{\nu}_{\mu \nu},
\end{equation}
la expresi\'{o}n \eqref{eq:varSRicci} se reescribe como $g^{\alpha \beta} \delta R_{\alpha \beta} = \nabla_{\lambda} v^{\lambda}$. Aplicando el teorema de Gauss se obtiene
%
\begin{equation}
\label{eq:intDv}
\int_{\Omega} g^{\mu \nu} \left(  \delta R_{\mu \nu} \right) \sqrt{-g} d^4 x = \int_{\Omega} \nabla_{\mu} v^{\mu} \sqrt{-g} d^4 x = \oint_{\partial \Omega} \varepsilon v^{\mu} n_{\mu} \sqrt{|h|} d^3 y \, .
\end{equation}
%
Recordando la condici\'{o}n de frontera \eqref{eq:boundcond}, la variaci\'{o}n de los s\'{i}mbolos de Christoffel en $\partial \Omega$ es
%
\begin{equation}
\label{eq:varGamma}
\delta \Gamma^{\sigma}_{\mu \nu} = \frac{1}{2} g^{\sigma \lambda} (\partial_{\mu} \delta g_{\nu \lambda} + \partial_{\nu} \delta g_{\lambda \mu} - \partial_{\lambda} \delta g_{\mu \nu}),
\end{equation}
%
entonces sustituyendo \eqref{eq:varGamma} en \eqref{eq:v} se llega a que
%
\begin{align}
\label{eq:vn}
n^{\lambda} v_{\lambda} \Big|_{\partial \Omega} & = n^{\lambda} g^{\mu \nu} (\partial_{\mu} \delta g_{\lambda \nu} - \partial_{\lambda} \delta g_{\mu \nu}) \nonumber \\
& = n^{\lambda} (\varepsilon n^{\mu} n^{\nu} + h^{\mu \nu}) (\partial_{\mu} \delta g_{\lambda \nu} - \partial_{\lambda} \delta g_{\mu \nu}) \nonumber \\
& = n^{\lambda} h^{\mu \nu} (\partial_{\mu} \delta g_{\lambda \nu} - \partial_{\lambda} \delta g_{\mu \nu}).
\end{align}
%
En \eqref{eq:vn} se uso la relaci\'{o}n de completez $g^{\mu \nu} = \varepsilon n^{\mu} n^{\nu} + h^{\mu \nu}$. Ahora, por la condici\'{o}n de frontera \eqref{eq:boundcond}, las derivadas tangentes $e^{\alpha}_{\beta} \partial_{\alpha} \delta g_{\mu \nu}$ a $\partial \Omega$ deben anularse, sin embargo las derivadas normales de $\delta g_{\mu \nu}$ en $\partial \Omega$ no necesariamente se anulan, por lo que

\begin{equation}
\label{eq:vnh}
n^{\mu} v_{\mu} = - n^{\mu} h^{\alpha \beta} \partial_{\mu} \delta g_{\alpha \beta}.
\end{equation}
%
Reemplazando \eqref{eq:vnh} en \eqref{eq:intDv}, se llega a
%
\begin{equation}
\label{eq:intvnh}
\int_{\Omega} g^{\mu \nu} (\delta R_{\mu \nu}) \sqrt{-g} d^4 x = - \oint_{\partial \Omega} \varepsilon n^{\mu} h^{\alpha \beta} \partial_{\mu} \delta g_{\alpha \beta} \sqrt{|h|} \, d^{3} y,
\end{equation}
%
y finalmente, sustituyendo \eqref{eq:intvnh} en \eqref{eq:IEH} se obtiene que la variaci\'{o}n del t\'{e}rmino de Einstein-Hilbert $I_{EH}$ est\'{a} dada por
%
\begin{equation}
\label{eq:varIEH}
\delta I_{EH} = \int_{\Omega} \left(R_{\mu \nu} - \frac{1}{2} R g_{\mu \nu} \right) \sqrt{-g} \delta g^{\mu \nu} \, d^{4} x - \oint_{\partial \Omega} \varepsilon n^{\mu} h^{\alpha \beta} \partial_{\mu} \delta g_{\alpha \beta} \sqrt{|h|} \, d^{3} y.
\end{equation}

Ahora se hace la variaci\'{o}n de $I_{B}$. N\'{o}tese que la \'{u}nica cantidad que var\'{i}a es la traza de la curvatura extr\'{i}nseca, i.e. $K$, pues la m\'{e}trica inducida en $\partial \Omega$ se mantiene fija, es decir, $\delta \sqrt{|h|} = 0$, as\'{i} que s\'{o}lo es necesario calcular la variaci\'{o}n de $K$ en $\partial \Omega$. Puesto que,
%
\begin{align}
\label{eq:definitionK}
K & = \nabla_{\mu} n^{\mu} = (\varepsilon n^{\mu} n^{\nu} + h^{\mu \nu}) \nabla_{\mu} n_{\nu} \nonumber \\
& = h^{\mu \nu} \nabla_{\mu} n_{\nu} = h^{\mu \nu} (\partial_{\mu} n_{\nu} - \Gamma^{\lambda}_{\mu \nu} n_{\lambda}),
\end{align}
%
usando lo que ya se conoce, \eqref{eq:varGamma} y que las derivadas tangentes $e^{\alpha}_{\beta} \partial_{\alpha} \delta g_{\mu \nu}$ se anulan, se llega a que
%
\begin{equation}
\label{eq:varK}
\delta K = -h^{\mu \nu} \delta \Gamma^{\lambda}_{\mu \nu} n_{\lambda} = \frac{1}{2} n^{\alpha} h^{\mu \nu} \partial_{\alpha} \delta g_{\mu \nu}.
\end{equation}
%
De esta manera se obtiene la variaci\'{o}n del t\'{e}rmino de frontera $\delta I_{B}$,
%
\begin{equation}
\label{eq:varIB}
2 \delta I_{B} [g] = \oint_{\partial \Omega} \varepsilon n^{\mu} h^{\alpha \beta} \partial_{\mu} \delta g_{\alpha \beta} \sqrt{|h|} \, d^{3} y.
\end{equation}

El t\'{e}rmino $I_{0}$ se escribe para tener un funcional de acci\'{o}n bien definido. Considerando un espacio-tiempo asint\'{o}ticamente plano, no compacto y vac\'{i}o, sin el t\'{e}rmino $I_{0}$ el funcional de acci\'{o}n $S$ tendr\'{a} un valor num\'{e}rico
%
\begin{equation}
\label{eq:SGI0}
S = \frac{1}{8 \pi} \oint_{\partial \Omega} \varepsilon K \sqrt{|h|} \, d^{3} y,
\end{equation}
%
el cual diverge como $I_{0}$ para espacios asint\'{o}ticamente planos, a\'{u}n y cuando $\Omega$ est\'{e} acotado por dos hipersuperficies espacialoides, entonces $S$ no estar\'{i}a bien definido. Sin embargo, al escribir el t\'{e}rmino $I_{0}$, se asegura que el funcional de acci\'{o}n para espacios asint\'{o}ticamente planos s\'{i} est\'{e} bien definido.

Dado que $I_{0}$ depende \'{u}nicamente de la m\'{e}trica inducida $h_{\alpha \beta}$, la variaci\'{o}n respecto a $g_{\mu \nu}$ es cero, por lo que $I_{0}$ no influye en las ecuaciones de campo al ser un t\'{e}rmino no din\'{a}mico. Como resultado de esto, la variaci\'{o}n del funcional de acci\'{o}n correspondiente a la gravedad, $S_{G} [g]$ es la suma de \eqref{eq:varIEH} m\'{a}s \eqref{eq:varIB}, por lo tanto
%
\begin{equation}
\label{eq:varSG}
\delta S_{G} [g] = \frac{1}{16 \pi} \int_{\Omega} \left( R_{\mu \nu} - \frac{1}{2} R g_{\mu \nu} \right) \sqrt{-g} \delta g^{\mu \nu} \, d^{4} x.
\end{equation}

Ya que el funcional de acci\'{o}n en Relatividad General est\'{a} formado por dos partes, $S_{G}$ y $S_{M}$, la variaci\'{o}n es la suma $\delta S = \delta S_{G} + \delta S_{M}$ (ecuaciones \eqref{eq:varSML} y \eqref{eq:varSG}), finalmente queda
%
\begin{equation}
\label{eq:varSSGSM}
\delta S = \int_{\Omega} \left( \frac{1}{16 \pi} (R_{\mu \nu} - \frac{1}{2} R g_{\mu \nu}) - \frac{1}{2} T_{\mu \nu} \right) \delta g^{\mu \nu} \sqrt{-g}.
\end{equation}

Y dado que la acci\'{o}n es extremizada, i.e. $\delta S = 0$, y $\delta g_{\mu \nu}$ son variaciones arbitrarias excepto por la condici\'{o}n \eqref{eq:boundcond}, las ecuaciones del campo gravitacional son
%
\begin{equation}
\label{eq:Einstein}
G_{\mu \nu} = 8 \pi T_{\mu \nu},
\end{equation}
%
donde $G_{\mu \nu} = R_{\mu \nu} - \frac{1}{2} R g_{\mu \nu}$ es el tensor de Einstein.
