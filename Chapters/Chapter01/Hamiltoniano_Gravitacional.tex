%---------------------------------------------------------------------------
% HAMILTONIANO GRAVITACIONAL
%---------------------------------------------------------------------------

\subsection{Hamiltoniano Gravitacional}

En el formalismo (3+1) el sistema es descrito por $h_{ab}$ y por c\'{o}mo \'{e}sta cambia a lo largo del \emph{flujo de tiempo}, es decir, del campo vectorial $t^{\alpha} = N n^{\alpha} + N^{\alpha}$. Las variables fundamentales son la m\'{e}trica $h_{ab}$ y su \emph{momento can\'{o}nicamente conjugado} $p^{ab}$,
%
\begin{equation}
p^{ab} = \frac{\partial \mathcal{L}_{G}}{\partial \dot{h}_{ab}}.
\end{equation}
%
Aqu\'{i}, $16 \pi \mathcal{L}_{G} = (^{(3)}R + K^{ab} K_{ab} - K^{2}) N \sqrt{h}$, y $\dot{h}_{ab}$ es el cambio de la m\'{e}trica a trav\'{e}s del flujo de tiempo, esto es,
%
\begin{equation}
\dot{h}_{ab} = \pounds_{t} h_{ab}.
\end{equation}

Recordando que $h_{ab} = g_{\alpha \beta} e^{\alpha}_{a} e^{\beta}_{b}$, se calcula la derivada de Lie de la m\'{e}trica $g_{\mu \nu}$ a lo largo del campo vectorial $t$, despu\'{e}s se proyecta en la hipersuperficie $\Sigma_{t}$ y se obtiene que
%
\begin{equation}
\label{eq:doth}
\dot{h}_{ab} = 2 N K_{ab} + D_{b} N_{a} + D_{a} N_{b},
\end{equation}
%
donde $D_{a}$ es la derivada covariente en $\Sigma_{t}$ combatible con $h_{ab}$.
%\begin{equation}
%\dot{h}_{ab} = \pounds_{t} (g_{\alpha \beta} e^{\alpha}_{a} e^{\beta}_{b}) = (\pounds_{t} g_{\alpha %\beta}) e^{\alpha}_{a} e^{\beta}_{b},
%\end{equation}
%
%\begin{align}
%\pounds_{t} g_{\alpha \beta} & = \nabla_{\beta} t_{\alpha} + \nabla_{\alpha} t_{\beta} \nonumber \\
%& = \nabla_{\beta} (N n_{\alpha} + N_{\alpha}) + \nabla_{\alpha} (N n_{\beta} + N_{\beta}) \nonumber \\
%& = n_{\alpha} \partial_{\beta} N + n_{\beta} \partial_{\alpha} N + N (\nabla_{\beta} n_{\alpha} + \nabla_{\alpha} n_{\beta}) + \nabla_{\beta} N_{\alpha} + \nabla_{\alpha} N_{\beta}.
%\end{align}
%
Luego se despeja la curvatura extr\'{i}nseca de \eqref{eq:doth}, para poder ver la dependencia en $\dot{h}_{ab}$ del funcional de acci\'{o}n gravitacional, entonces,
%
\begin{equation}
\label{eq:Kab(doth)}
K_{ab} = \frac{1}{2 N} \left( \dot{h}_{ab} - D_{b} N_{a} - D_{a} N_{b} \right).
\end{equation}

De la ecuaci\'{o}n anterior, \eqref{eq:Kab(doth)}, n\'{o}tese que solamente la parte de volumen de la densidad Lagrangiana de gravedad depende de $\dot{h}_{ab}$ (ver \eqref{eq:LagSg}), as\'{i} que para facilitar las cosas, se analizar\'{a} s\'{o}lo esa parte, entonces
%
\begin{align}
\label{eq:pabADM}
(16 \pi) p^{ab} & = \frac{\partial K_{mn}}{\partial \dot{h}_{ab}} \frac{\partial}{\partial K_{mn}} \left( 16 \pi \mathcal{L}_{G} \right) \nonumber \\
& = \frac{\partial K_{mn}}{\partial \dot{h}_{ab}} \frac{\partial}{\partial K_{mn}} \left( [^{(3)}R + (h^{ac} h^{bd} - h^{ab} h^{cd}) K_{ab} K_{cd}] N \sqrt{h} \right) \nonumber \\
& = \sqrt{h} (K^{ab} - K h^{ab}).
\end{align}

Haciendo la transformaci\'{o}n de Legendre $\mathcal{H}_{G} = p^{ab} \dot{h}_{ab} - \mathcal{L}_{G}$, la parte de volumen de la densidad Hamiltoniana es
%
\begin{align}
16 \pi \mathcal{H}_{G} & = \sqrt{h} (K^{ab} - K h^{ab}) (2 N K_{ab} + D_{b} N_{a} + D_{a} N_{b}) \nonumber \\
& \quad - (^{(3)}R + K^{ab} K_{ab} - K^{2}) N \sqrt{h} \nonumber \\
& = (K^{ab} K_{ab} - K^{2} - ^{(3)}R) N \sqrt{h} + 2 (K^{ab} - K h^{ab}) D_{b} N_{a} \sqrt{h} \nonumber \\
& = (K^{ab} K_{ab} - K^{2} - ^{(3)}R) N \sqrt{h} + 2 D_{b} [(K^{ab} - K h^{ab}) N_{a}] \sqrt{h} \nonumber \\
& \quad - 2 D_{b} (K^{ab} - K h^{ab}) N_{a} \sqrt{h}.
\end{align}

El Hamiltoniano gravitacional lo obtenemos integrando $\mathcal{H}_{G}$ sobre $\Sigma_{t}$ y sumando los t\'{e}rminos de frontera,
%
\begin{align}
\label{eq:HG}
16 \pi H_{G} & = \int_{\Sigma_{t}} 16 \pi \mathcal{H}_{G} \, d^{3} y - 2 \oint_{S_{t}} (k - k_{0}) N \sqrt{\sigma} \, d^{2} \theta \nonumber \\
& = \int_{\Sigma_{t}} \left[ N (K^{ab} K_{ab} - K^{2} - ^{(3)}R) - 2 N_{a} D_{b} (K^{ab} - K h^{ab}) \right] \sqrt{h} \, d^{3} y \nonumber \\
& \quad + 2 \oint_{S_{t}} (K^{ab} - K h^{ab}) N_{a} d S_{b} - 2 \oint_{S_{t}} (k - k_{0}) N \sqrt{\sigma} d^{2} \theta \nonumber \\
& = \int_{\Sigma_{t}} \left[ N (K^{ab} K_{ab} - K^{2} - ^{(3)}R) - 2 N_{a} D_{b} (K^{ab} - K h^{ab}) \right] \sqrt{h} \, d^{3} y \nonumber \\
& \quad - 2 \oint_{S_{t}} \left[ (k - k_{0}) N - (K^{ab} - K h^{ab}) N_{a} r_{b} \right] \sqrt{\sigma} d^{2} \theta,
\end{align}
%
donde $d S_{b} = r_{b} \sqrt{\sigma} d^{2} \theta$.

En el Hamiltoniano, ecuaci\'{o}n \eqref{eq:HG}, $K^{ab}$ representa a las funciones $h^{ab}$ y $p^{ab}$, expl\'{i}citamente
%
\begin{equation}
\label{eq:Kph}
\sqrt{h} K^{ab} = 16 \pi ( p^{ab} - \frac{1}{2} p h^{ab}).
\end{equation}
%
con $p:= h_{ab} p^{ab}$.