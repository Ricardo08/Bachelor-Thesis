\section{Espacio fase}

Comencemos considerando un campo de cotriadas $e^{I}_{a}$ ($I=1,2,3$ y $a=1,2,3$) sobre la variedad  tridimensional $\Sigma_{t}$ (secci\'{o}n \ref{subsec:3+1}). La m\'{e}trica $h_{ab}$ en $\Sigma_{t}$ puede expresarse como\footnotemark
\footnotetext{Dado que aqu\'{i} el espacio interno es $\mathbb{R}^{3}$, la m\'{e}trica interna es $\delta_{IJ}$.}
%
\begin{equation}
\label{eq:3metrictriad}
h_{ab} = \delta_{IJ} e^{I}_{a} e^{J}_{b}.
\end{equation}
%
Donde se observa que \eqref{eq:3metrictriad} es invariante bajo transformaciones de $\mathrm{SO}(3)$, ya que $\delta_{IJ} e'^{I}_{a} e'^{J}_{b} = \delta_{IJ} e^{I}_{a} e^{J}_{b}$ donde $e'^{I}_{a} = O^{I}_{J} e^{J}_{a}$ y $O^{I}_{J}$ es un elemento de matriz de $\mathrm{SO}(3)$. Las triadas y cotriadas \emph{densitizadas}, $E^{a}_{I}$ y $E^{I}_{a}$, son (respectivamente):
%
\begin{equation*}
E^{a}_{I} = \sqrt{h} e^{a}_{I} \quad \mathrm{y} \quad E^{I}_{a} = e^{I}_{a}/\sqrt{h}.
\end{equation*}
%
Expresando el determinante de $h$ como $|\det(E)|$, la variable can\'{o}nica $h_{ab}$ de la formulaci\'{o}n ADM queda como
%
\begin{equation}
\label{eq:habEE}
h_{ab} = |\det(E)| \delta_{IJ} E^{I}_{a} E^{J}_{b}
\end{equation}
%
y su inverso
%
\begin{equation}
h^{ab} = |\det(E)|^{-1} \delta^{IJ} E^{a}_{I} E^{b}_{J}.
\end{equation}

Ahora, para reescribir el momento can\'{o}nicamente conjugado $p^{ab}$ de la formulaci\'{o}n ADM \eqref{eq:pabADM} en t\'{e}rminos de las triadas, se debe considerar la curvatura extr\'{i}nseca $K_{ab}$ de $\Sigma_{t}$. Sea la una 1-forma $K_{aI}$ en $\Sigma_{t}$ tal que
%
\begin{equation}
K_{ab} = K_{aI} e^{I}_{b},
\end{equation}
%
con lo que se sigue que el momento can\'{o}nicamente conjugado en \eqref{eq:pabADM} se expresa como
%
\begin{equation}
\label{eq:pabEE}
(16 \pi)p^{ab} = 2 |\det(E)|^{-1} \delta^{IJ} E^{a}_{I} E^{d}_{J} K^{K}_{[d} \delta^{b}_{c]} E^{c}_{K}.
\end{equation}

Adem\'{a}s, como $K_{ab} = K_{ba}$, se tienen tres constricciones:
%
\begin{equation}
\label{eq:Gab0}
\mathcal{G}_{ab} \equiv K_{[ab]} = K_{[aI} e^{I}_{b]} = 0.
\end{equation}
%
Si se contrae $\mathcal{G}_{ab}$ con $e^{a}_{I}$ y $e^{b}_{J}$ se llega a que
%
\begin{align*}
\mathcal{G}_{ab} e^{a}_{I} e^{b}_{J} & = \frac{1}{2}(K_{aK} e^{K}_{b} - K_{bK} e^{K}_{a}) e^{a}_{I} e^{b}_{J} = \frac{1}{2} (K_{aK} e^{a}_{I} \delta^{K}_{J} - K_{bK} e^{b}_{J} \delta^{K}_{I}) \\
& = \frac{1}{2} (K_{aJ} e^{a}_{I} - K_{bI} e^{b}_{J}) = K_{a[J} e^{a}_{I]} \\
& \qquad \qquad \Longrightarrow K_{a[J} e^{a}_{I]} = 0,
\end{align*}
%
es decir, las constricciones \eqref{eq:Gab0} se pueden reescribir como $h^{-1/2} K_{a[J} E^{a}_{I]} = 0$ o de mejor manera
%
\begin{equation}
\label{eq:GJI0}
\mathcal{G}_{JI} \equiv K_{a[J} E^{a}_{I]} = 0.
\end{equation}

Usando las expresiones \eqref{eq:habEE} y \eqref{eq:pabEE}, las constricciones Hamiltoniana y de difeormorfismos \eqref{eq:constrictions}, quedan como
%
\begin{equation}
\label{eq:CEE}
\mathcal{C} = \frac{-1}{16 \pi} [|\det(E)|^{-1/2} (K^{I}_{a} K^{J}_{b} - K^{J}_{a} K^{I}_{b}) E^{a}_{I} E^{b}_{J} + |\det(E)|^{1/2} \, ^{(3)}R]
\end{equation}
%
\begin{equation}
\label{eq:CaEE}
\mathcal{C}_{a} = -2 D_{b} (K^{J}_{a} E^{b}_{J} - \delta^{b}_{a} K^{J}_{c} E^{c}_{J}),
\end{equation}
%
donde $^{(3)}R$ es considerado como una funci\'{o}n de las triadas densitizadas $E^{a}_{I}$.

En la formulaci\'{o}n Hamiltoniana ADM, hay seis variables de configuraci\'{o}n ($h_{ab}$) y sus seis respectivos momentos can\'{o}nicos ($p^{ab}$), sujetos a cuatro constricciones, una Hamiltoniana ($\mathcal{C} = 0$) y tres de difeormorfismos ($\mathcal{C}^{a} = 0$); dando un total de dos grados f\'{i}sicos de libertad\footnotemark. Sin embargo, al extender el espacio fase con las nuevas variables $(K^{I}_{a}, E^{b}_{J})$, el n\'{u}mero de grados de libertad aument\'{o} de 12 a 18, pero el n\'{u}mero de grados f\'{i}sicos de libertad debe seguir siendo el mismo as\'{i} que se toman en cuenta las tres constricciones $\mathcal{G}_{IJ} = 0$ m\'{a}s las cuatro ya mencionadas; conservando los dos grados f\'{i}sicos de libertad.
\footnotetext{Conteo de grados f\'{i}sicos de libertad:
\begin{center}
2(n\'{u}m. de grados f\'{i}sicos de libertad) = (n\'{u}m. de variables can\'{o}nicas) - 2 (n\'{u}m. de constricciones de primera clase).
\end{center}}

Lo siguiente por hacer, es revisar la estructura simpl\'{e}ctica del espacio fase extendido $(K^{I}_{a}, E^{b}_{J})$.
%
\begin{equation}
\{E^{a}_{I}(y), K^{J}_{b}(y')\} = \frac{1}{2} \delta^{a}_{b} \delta^{I}_{J} \delta(y-y')
\end{equation}
%
\begin{equation}
\{E^{a}_{I}(y), E^{b}_{J}(y')\} = \{K^{I}_{a}(y), K^{J}_{b}(y')\} = 0.
\end{equation}
%
As\'{i}, siempre que se cumpla la condici\'{o}n $\mathcal{G}_{IJ} = 0$, los par\'{e}ntesis de Poisson de las variables can\'{o}nicas $(h_{ab}, p^{ab})$ como funciones de $K^{I}_{a}$ y $E^{a}_{J}$ en el espacio fase extendido dan
%
\begin{equation}
\{h_{ab}(y), p^{ab}(y')\} = \delta^{c}_{(a} \delta^{d}_{b)} \delta(y-y')
\end{equation}
%
\begin{equation}
\{h_{ab}(y), h_{cd}(y')\} = \{p^{ab}(y), p^{cd}(y')\} = 0,
\end{equation}
%
que son los par\'{e}ntesis de Poisson del espacio fase ADM. Por lo tanto, si la condici\'{o}n $\mathcal{G}_{IJ} = 0$ se satisface entonces, las variables can\'{o}nicas ADM expresadas como funciones de las variables $(K^{I}_{a}, E^{b}_{J})$ del espacio fase extendido generan los mismos par\'{e}ntesis de Poisson.

Ahora consideremos la constricci\'{o}n \eqref{eq:GJI0} de la forma siguiente
%
\begin{equation}
\mathcal{G}(\Lambda) = \int\limits_{\Sigma_{t}} \Lambda^{IJ} K_{aI} E^{a}_{J} d^{3}y,
\end{equation}
%
con $\Lambda^{IJ}$ siendo las componentes de una matriz antisim\'{e}trica que genera rotaciones $\mathrm{SO}(3)$. Calculando el par\'{e}ntesis de Poisson consigo mismo se llega a que
%
\begin{equation}
\{\mathcal{G}(\Lambda), \mathcal{G}(\Lambda')\} = \frac{1}{2} \mathcal{G}([\Lambda, \Lambda']),
\end{equation}
%
que es igual al \'{a}lgebra de rotaciones espaciales $\mathrm{SO}(3)$. Y dado que las variables can\'{o}nicas de ADM son invariantes bajo $\mathrm{SO}(3)$, cualquier par\'{e}ntesis de Poisson entre ellas y $G(\Lambda)$ se anula.

En t\'{e}rminos de las nuevas variables, la acci\'{o}n gravitacional es
%
\begin{equation}
S_{G}[K,E] = \frac{1}{16 \pi} \int\limits^{t_{2}}_{t_{1}} dt \int\limits_{\Sigma_{t}} \left[2 \dot{K}^{K}_{a} E^{a}_{K} - (\Lambda^{IJ} \mathcal{G}_{IJ} + \mathcal{C}_{a} N^{a} + \mathcal{C} N) \right] d^{3}y
\end{equation}
%
donde las constricciones est\'{a}n dadas por \eqref{eq:GJI0}, \eqref{eq:CEE} y \eqref{eq:CaEE}. N\'{o}tese que esta acci\'{o}n es equivalente a la del formalismo 3+1 siempre y cuando se cumpla la constricci\'{o}n $\mathcal{G}_{IJ} = 0$.
