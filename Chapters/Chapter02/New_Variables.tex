Al realizar la formulaci�n Hamiltoniana a trav�s del formalismo ADM se obtienen las variables can�nicas $h_{ab}$ y $p^{ab}$ que est�n sujetas a cuatro constricciones $\mathcal{C}=0$ y $\mathcal{C}^{a}=0$ ($a=1,2,3$) \eqref{eq:constrictions}. Estas constricciones tienen una complicada dependencia no polinomial de las variables can�nicamente conjugadas, lo que ha hecho pr�cticamente imposible construir la representaci�n de momento en los intentos de cuatizaci�n con estas variables. Sin embargo, a mediados de los 80's A. Ashtekar not� que el espacio fase gravitacional podr�a reformularse en t�rminos de un nuevo par de variables can�nicas que simplifican las constricciones \cite{Ashtekar86, Ashtekar87}. Estas variables son ciertas conexiones de esp�n modificadas de aquellas construidas por A. Sen para los espinores $SL(2, \mathbb{C})$ \cite{Sen1, Sen2} y su momento can�nicamente conjugado. En esta nueva formulaci�n de la Relatividad General, en t�rminos de la conexi�n modificada y de su momento conjugado, se simplifica notablemente la forma de las constricciones y con ello la b�squeda de una teor�a cu�ntica de la Relatividad General. Adem�s, algo muy importante por destacar de esta nueva formulaci�n Hamiltoniana, es que la estructura matem�tica de la teor�a de Einstein se ve como una teor�a tipo Yang-Mills.

Aunque en esta formulaci�n las constricciones tienen una forma m�s sencilla de manejar, las variables fundamentales son complejas, lo que gener� problemas con la implementaci�n de las \emph{condiciones de realidad}. Para lidiar con esto, J. F. Barbero propuso una variante en la cual la conexi�n es real \cite{Barbero}, y es conocida como la conexi�n de Ashtekar-Barbero. La relaci�n entre las variables reales y complejas es en cierta medida clarificada por el par�metro de Barbero-Immirzi, $\beta$; las nuevas variables corresponden a las (anti-)auto-duales para $\beta = \pm i$ y a las reales para cualquier $\beta \in \mathbb{R}$ \cite{Immirzi, RovelliThiemann}.
%
%Para incorporar espinores a la teor�a se debe extender el espacio fase. Ashtekar originalmente se bas� en el c�lculo espinorial $\mathrm{SL}(2,\mathbb{C})$, pero aqu� la extensi�n del espacio fase se har� mediante el uso de triadas \cite{Vega}, pues permite ver de manera m�s clara la equivalencia entre la nueva formulaci�n Hamiltoniana y la formulaci�n ADM. Despu�s de extender el espacio fase de ADM, se implementa una transformaci�n can�nica y as�, obtener la formulaci�n de la Relatividad General en variables de conexi�n.