\section{Transformaci\'{o}n Can\'{o}nica}

Como ya se mencion\'{o}, al extender el espacio fase se ganan m\'{a}s grados de libertad pero para conservar el n\'{u}mero de grados f\'{i}sicos de libertad tambi\'{e}n aparecieron m\'{a}s constricciones. El prop\'{o}sito a continuaci\'{o}n es escribir estas constricciones (ecuaci\'{o}n \eqref{eq:GJI0}) como una ley de Gauss, an\'{a}logamente a como es en electromagnetismo y la teor\'{i}a de Yang-Mills, que es justo lo que se desea al formular la Relatividad General en t\'{e}rminos de Variables de Conexi\'{o}n. Esto es escribir $\mathcal{G}_{IJ} = 0$ como $(\partial_{a} E^{a} + [A_{a}, E^{a}])_{IJ}$ donde $A$ es alguna conexi\'{o}n con valores en el \'{a}lgebra de Lie de $\mathrm{SO}(3)$. Para ello se realiza una transformaci\'{o}n can\'{o}nica en el espacio fase $(K^{I}_{a}, E^{b}_{J})$.

Antes de continuar se introducir\'{a}n unos conceptos que ser\'{a}n de ayuda.

Se define la derivada covariente $D_{a}$ en la hipersuperficie $\Sigma_{t}$ sobre un objeto $v_{I}$ como
%
\begin{equation}
D_{a} v_{I} = \partial_{a} v_{I} + w_{aI}\,^{K} v_{K}
\end{equation}
%
donde $w_{aI}\,^{K}$ son las componentes de la \emph{conexi\'{o}n de esp\'{i}n} $w$ del \'{a}lgebra de Lie de $\mathrm{SO}(3)$. As\'{i} se tiene que la acci\'{o}n sobre las triadas est\'{a} dada por
%
\begin{equation}
\label{eq:DaebI}
D_{a} e^{b}_{I} = \partial_{a} e^{b}_{I} + \Gamma^{b}_{ac} e^{c}_{I} + w_{aI}\,^{K} e^{b}_{K}.
\end{equation}
%
Observando que $w_{aI}\,^{K} y^{a} \in \mathrm{SO}(3)$, el potencial $w_{aI}\,^{K}$ es una 1-forma de conexi\'{o}n valuada en el \'{a}lgebra de Lie de $\mathrm{SO}(3)$, lo que implica que $w_{a(I}\,^{K)} = 0$.

Ahora, a partir de la identidad $e^{a}_{I} e^{J}_{b} = \delta^{J}_{I}$ se tiene que
%
\begin{align*}
0 & = D_{a} (e^{b}_{I} e^{J}_{b}) = (D_{a} e^{b}_{I}) e^{J}_{b} + e^{b}_{I} (D_{a} e^{J}_{b}) \\
& \Longrightarrow e^{b}_{I} (D_{a} e^{J}_{b}) = -(D_{a} e^{b}_{I}) e^{J}_{b}
\end{align*}
%
Contrayendo con $e^{I}_{c}$ y tomando en cuenta la ecuaci\'{o}n \eqref{eq:DaebI} se llega a que
%
\begin{align*}
D_{a} e^{J}_{c} & = -(D_{a} e^{b}_{I}) e^{I}_{c} e^{J}_{b} \\
& = -(\partial_{a} e^{b}_{I} + \Gamma^{b}_{ac} e^{c}_{I} + w_{aI}\,^{K} e^{b}_{K}) e^{I}_{c} e^{J}_{b} \\
& = -(\partial_{a} e^{b}_{I}) e^{I}_{c} e^{J}_{b} - \Gamma^{b}_{ac} e^{J}_{b} - w_{aI}\,^{J} e^{K}_{c}.
\end{align*}
%
Y de las igualdades $\partial_{a}(e^{I}_{c}) = \partial_{a}(e^{b}_{K} e^{K}_{c} e^{I}_{b}) = \partial_{a}(e^{b}_{K}) e^{K}_{c} e^{I}_{b} + 2 \partial_{a}(e^{I}_{c})$ se sigue que
%
\begin{equation}
D_{a} e^{J}_{c} = \partial_{a} e^{J}_{c} - \Gamma^{b}_{ac} e^{J}_{b} - w_{aK}\,^{J} e^{K}_{c}.
\end{equation}
%
Finalmente, esto lleva a que la derivada covariante espacial sobre un objeto $v^{I}$ es
%
\begin{equation}
D_{a} v^{I} = \partial_{a} v^{I} - w_{aK}\,^{I} v^{K} = \partial_{a} v^{I} + w_{a}\,^{I}\,_{K} v^{K}.
\end{equation}

Ahora bien, la transformaci\'{o}n can\'{o}nica a realizar consiste de un rescalamiento y una transformaci\'{o}n af\'{i}n. Considerando el rescalamiento
%
\begin{equation}
(K^{I}_{a}, E^{b}_{J}) \longmapsto (\tilde{K}^{I}_{a}, \tilde{E}^{b}_{I}) = (\beta K^{I}_{a}, \beta^{-1} E^{b}_{J}),
\end{equation}
%
donde $\beta \in \mathbb{R}/\{0\}$ es el par\'{a}metro de Barbero-Immirzi, las tres constricciones $\mathcal{G}_{IJ} = 0$, que conviene reescribirlas como $\mathcal{G}_{K} = \epsilon_{KIJ} \delta^{JL} K^{I}_{a} E^{a}_{L}$, se mantienen invariantes bajo el rescalamiento,
%
\begin{equation}
\mathcal{G}_{K} = \epsilon_{KI}\,^{J} \tilde{K}^{I}_{a} \tilde{E}^{a}_{J}.
\end{equation}

Continuemos con la transformaci\'{o}n af\'{i}n. Recurriendo a la ecuaci\'{o}n \eqref{eq:DaebI} pero \emph{densitizada} y dado que $D_{a}$ debe ser compatible con $h_{ab}$ por lo que $D_{a} e^{b}_{I} = 0$ y $D_{a} \sqrt{h} = 0$; tenemos que la divergencia de $E^{a}_{I}$ es
%
\begin{align*}
D_{a} E^{a}_{I} & = \sqrt{h} \partial_{a} e^{a}_{I} + \Gamma^{a}_{ac} E^{c}_{I} + w_{aJ}\,^{K} E^{a}_{K} \\
& = \sqrt{h} \partial_{a} e^{a}_{I} + \frac{1}{\sqrt{h}} (\partial_{c} \sqrt{h}) e^{c}_{I} + w_{aI}\,^{K} E^{a}_{K} \\
& = \partial_{a} E^{a}_{I} + w_{aI}\,^{K} E^{a}_{K}.
\end{align*}

Por otro lado, de $D_{a} e^{I}_{b} = 0$, la conexi\'{o}n de esp\'{i}n se puede escribir como
%
\begin{equation}
\label{eq:waKI1}
w_{aK}\,^{I} = e^{b}_{K} (\partial_{a} e^{I}_{b} - \Gamma^{c}_{ab} e^{I}_{c}),
\end{equation}
%
pero recordando que $\Gamma^{c}_{ab} = 1/2 h^{cd} (\partial_{a} h_{bd} + \partial_{b} h_{da} - \partial_{d} h_{ab})$, $h_{ab} = \delta_{IJ} e^{I}_{a} e^{J}_{b}$ y $h^{ab} = \delta^{IJ} e^{a}_{I} e^{b}_{J}$, despu\'{e}s de varias cuentas, la expresi\'{o}n \eqref{eq:waKI1} se reescribe como
%
\begin{align}
\label{eq:waKI2}
w_{aK}\,^{I} & = 2 e^{d}_{[K} \partial_{[a} e^{I]}_{d]} + e^{J}_{a} e^{d}_{K} e^{bI} \partial_{[b} e_{d]J} \nonumber \\
& = e^{d[I} \partial_{d} e_{aK]} - e^{d[I} \partial_{a} e_{dK]} + e^{J}_{a} e^{d}_{[K} e^{bI]} \partial_{b} e_{dJ}.
\end{align}
%
La cual puede reducirse bajando el \'{i}ndice $I$ y usando las identidades
%
\begin{align*}
%e^{b[I} \partial_{b} e_{c}^{J]} & = \frac{1}{2} \epsilon_{KMN} \epsilon^{KJI} e^{bN} \partial_{b} e^{M}_{c} \\
e^{d}_{[I} \partial_{d} e_{aK]} & = \frac{1}{2} \epsilon_{JKI} \epsilon^{JMN} e^{d}_{N} \partial_{d} e_{aM} \\
e^{d}_{[I} \partial_{a} e_{dK]} & = \frac{1}{2} \epsilon_{JKI} \epsilon^{JMN} e^{d}_{N} \partial_{a} e_{dM} \\
e_{aJ} e^{d}_{[K} e^{b}_{I]} \partial_{b} e^{J}_{d} & = \frac{1}{2} \epsilon_{LKI} \epsilon^{LMN} e^{d}_{M} e^{b}_{N} e_{aJ} \partial_{b} e^{J}_{d},
\end{align*}
%
entonces \eqref{eq:waKI2} queda como
%
\begin{equation}
w_{aKI} = \frac{1}{2} \epsilon_{JKI} \epsilon^{JMN} e^{d}_{N} (\partial_{d} e_{aM} - \partial_{a} e_{dM} + e^{b}_{M} e_{aR} \partial_{d} e^{R}_{b}).
\end{equation}
%
Luego si se define
%
\begin{equation}
w^{J}_{a} := -\frac{1}{2} \epsilon^{JMN} e^{d}_{N} (\partial_{d} e_{aM} - \partial_{a} e_{dM} + e^{b}_{M} e_{aR} \partial_{d} e^{R}_{b}),
\end{equation}
%
la conex\'{i}on de esp\'{i}n es
%
\begin{equation}
\label{eq:waKI}
w_{aKI} = -\epsilon_{JKI} w^{J}_{a}.
\end{equation}

Substituyendo \eqref{eq:waKI} en en la divergencia de $E^{a}_{I}$ se tiene que
%
\begin{equation}
\label{eq:DaEaI}
D_{a} E^{a}_{I} = \partial_{a} E^{a}_{I} - \epsilon_{JI}\,^{K} w^{J}_{a} E^{a}_{K} = \partial_{a} E^{a}_{I} + \epsilon_{IJ}\,^{K} w^{J}_{a} E^{a}_{K}.
\end{equation}

Dando seguimiento a la transformaci\'{o}n can\'{o}nica, ahora se expresa a $w^{J}_{a}$ en t\'{e}rminos de las triadas y cotriadas densitizadas ($E$'s) ya que hasta ahora est\'{a} en funci\'{o}n de las triadas y cotriadas ($e$'s).
%
\begin{align}
\label{eq:wEs}
w^{J}_{a} & = -\frac{1}{2} \epsilon^{JIK} E^{b}_{K} [\partial_{b} E_{aI} - \partial_{a} E_{bI} + E^{c}_{I} E_{aL} \partial_{b} E^{L}_{c}] \nonumber \\
& \quad - \frac{1}{4 |\det(E)|} \epsilon^{JIK} E^{b}_{K} [2 E_{aI} \partial_{b} (|\det(E)|) - E_{bI} \partial_{a} (|\det(E)|)].
\end{align}
%
De \eqref{eq:wEs}, se debe notar que $w^{J}_{a}$ es invariante bajo el rescalamiento, as\'{i} que
%
\begin{equation}
\tilde{w}^{J}_{a} = w^{J}_{a} (\tilde{E}) = w^{J}_{a} (E).
\end{equation}

Retomando que la derivada covariante espacial debe ser compatible con la m\'{e}trica de $\Sigma_{t}$, implica $D_{a}(E^{b}_{I}) = 0$. Entonces $D_{a}(\tilde{E}^{a}_{I}) = 0$, pues $\beta$ es un escalar. Por lo tanto de la divergencia \eqref{eq:DaEaI} se sigue que
%
\begin{equation}
D_{a} \tilde{E}^{a}_{I} = \partial_{a} \tilde{E}^{a}_{I} + \epsilon_{IJ}\,^{K} w^{J}_{a} \tilde{E}^{a}_{K} = 0
\end{equation}

Con esto se escribe la constricci\'{o}n $\mathcal{G}_{I} = 0$ como
%
\begin{align}
\label{eq:GI0}
\mathcal{G}_{I} & = \partial_{a} \tilde{E}^{a}_{I} + \epsilon_{IJ}\,^{K} (w^{J}_{a} + \tilde{K}^{J}_{a} )\tilde{E}^{a}_{K} \nonumber \\
& \equiv \partial_{a} \tilde{E}^{a}_{I} + \epsilon_{IJ}\,^{K} \tilde{A}^{J}_{a} \tilde{E}^{a}_{K}.
\end{align}
%
El t\'{e}rmino $\tilde{A}^{J}_{a}$ es la conexi\'{o}n de Ashtekar-Barbero y \eqref{eq:GI0} es justamente la forma que se buscaba, que es an\'{a}loga a una constricci\'{o}n de Gauss de una teor\'{i}a de norma $\mathrm{SU}(2)$. Expl\'{i}citamente
%
\begin{equation}
\mathcal{G}_{I} = \tilde{D}_{a} \tilde{E}^{a}_{I} = \partial_{a} \tilde{E}^{a}_{I} + \epsilon_{IJ}\,^{K} \tilde{A}^{J}_{a} \tilde{E}^{a}_{K} = 0.
\end{equation}

Es sencillo verificar que despu\'{e}s de la transformaci\'{o}n
%
\begin{equation}
(K^{I}_{a}, E^{b}_{J}) \longmapsto (\tilde{A}^{I}_{a}, \tilde{E}^{b}_{J}),
\end{equation}
%
los par\'{e}ntesis de Poisson son
%
\begin{equation}
\label{eq:PoissonEA}
\{\tilde{E}^{b}_{J}(y), \tilde{A}^{I}_{a}(y')\} = \frac{1}{2} \delta^{b}_{a} \delta^{I}_{J} \delta(y-y')
\end{equation}
%
\begin{equation}
\{\tilde{E}^{a}_{I}(y), \tilde{E}^{b}_{J}(y')\} = \{\tilde{A}^{I}_{a}(y), \tilde{A}^{J}_{b}(y')\} = 0.
\end{equation}

Por \'{u}ltimo, nos queda expresar las constricciones \eqref{eq:CEE} y \eqref{eq:CaEE} en t\'{e}rminos de las nuevas variables $(\tilde{A}^{I}_{a}, \tilde{E}^{b}_{J})$ para concluir con la formulaci\'{o}n de la Relatividad General en variables de conexi\'{o}n.

Introduciendo las curvaturas asociadas a $w^{J}_{b}$ y a $A^{I}_{b}$ respectivamente
%
\begin{equation}
R^{J}_{ab} = 2 \partial_{[a} w^{J}_{b]} + \epsilon^{J}\,_{KL} w^{K}_{a} w^{L}_{b}
\end{equation}
%
\begin{equation}
\tilde{F}^{I}_{ab} = 2 \partial_{[a} \tilde{A}^{I}_{b]} + \epsilon^{I}\,_{KL} \tilde{A}^{K}_{a} \tilde{A}^{L}_{b}.
\end{equation}
%
Y escribiendo $\tilde{F}^{J}_{ab}$ en t\'{e}rminos de $w^{J}_{b}$ y $\beta K^{J}_{b}$ se obtiene una expresi\'{o}n que es posible relacionar m\'{a}s facilmente con las constricciones \eqref{eq:CEE} y \eqref{eq:CaEE}. \'{E}sta es de la forma
%
\begin{equation*}
\tilde{F}^{J}_{ab} = R^{J}_{ab} + 2 \beta D_{[a} K^{J}_{b]} + \beta^{2} \epsilon^{J}\,_{IK} K^{I}_{a} K^{K}_{b}.
\end{equation*}
%
Si se contrae con $\tilde{E}^{b}_{J}$ se obtiene
%
\begin{align*}
\tilde{F}^{J}_{ab} \tilde{E}^{b}_{J} & = R^{J}_{ab} \tilde{E}^{b}_{J} + 2 \beta D_{[a} K^{J}_{b]} \tilde{E}^{b}_{J} + \beta^{2} \epsilon^{J}\,_{IK} K^{I}_{a} K^{K}_{b} \tilde{E}^{b}_{J} \\
& = R^{J}_{ab} \tilde{E}^{b}_{J} + 2 \beta D_{[a} K^{J}_{b]} \tilde{E}^{b}_{J} + \beta K^{K}_{a} G_{K}.
\end{align*}
%
Es posible mostrar que el primer t\'{e}rmino se anula; mientras que si se cumple la constricci\'{o}n de Gauss el tercer t\'{e}rmino tambi\'{e}n se anula, entonces
%
\begin{equation*}
\tilde{F}^{J}_{ab} \tilde{E}^{b}_{J} = 2 \beta D_{[a} K^{J}_{b]} \tilde{E}^{b}_{J}.
\end{equation*}
%
Que es simplemente una manera de reescribir \eqref{eq:CaEE}, por lo tanto
%
\begin{equation}
\label{eq:CaEA}
\mathcal{C}_{a} = \tilde{F}^{J}_{ab} \tilde{E}^{b}_{J}.
\end{equation}

An\'{a}logamente, haciendo la contracci\'{o}n $\tilde{F}^{J}_{ab} \epsilon_{J}\,^{IK} \tilde{E}^{a}_{I} \tilde{E}^{b}_{K}$ y recurriendo al hecho de que $\mathcal{G}_{I} = 0$, se obtiene
%
\begin{equation}
\label{eq:CEA}
\mathcal{C} = \frac{\epsilon_{J}\,^{IK} \tilde{E}^{a}_{I} \tilde{E}^{b}_{K}}{\sqrt{|\det(\beta \tilde{E})|}} \left[\beta^{2} \tilde{F}^{J}_{ab} - (\beta^{2} - 1) \epsilon^{J}\,_{IK} \tilde{K}^{I}_{a} \tilde{K}^{K}_{b} \right].
\end{equation}

Viendo \eqref{eq:PoissonEA}, es decir, el par $(\tilde{A}^{I}_{a}, \tilde{E}^{b}_{J})$ son efectivamente variables can\'{o}nicamente conjugadas y considerando como se reescribieron las constricciones \eqref{eq:GI0}, \eqref{eq:CaEA} y \eqref{eq:CEA}; la acci\'{o}n de Relatividad General en variables de conexi\'{o}n queda como
%
\begin{equation}
S_{EH} = \frac{1}{16 \pi} \int\limits^{t_{2}}_{t_{1}} dt \int\limits_{\Sigma_{t}} \left[2 \dot{\tilde{A}}^{I}_{a} \tilde{E}^{a}_{I} - (\Lambda^{J} \mathcal{G}_{J} + N^{a} \mathcal{C}_{a} + N \mathcal{C}) \right] d^{3} y.
\end{equation}

La gran ventaja de esta formulaci\'{o}n se muestra en el aspecto cu\'{a}ntico, ya que es \emph{independiente de fondo}. Esto, la ha llevado a ser considerada como el punto de partida para una teor\'{i}a cu\'{a}ntica de la gravedad independiente de la m\'{e}trica.
