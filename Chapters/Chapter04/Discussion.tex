La Gravedad Cu\'{a}ntica de Lazos es un candidato para la cuantizaci\'{o}n de la gravedad a partir de los fundamentos de Relatividad General. Hasta ahora ha descubierto varias caracter\'{i}sticas cruciales relevantes en escalas de la longitud de Planck, ha logrado discretizar la geometr\'{i}a espacial en cu\'{a}ntos geom\'{e}tricos de longitud, \'{a}rea y volumen, que ha significado una gran ayuda en el estudio de las singularidades cosmol\'{o}gicas y de los agujeros negros. Sin embargo, a pesar de simplificar la contricci\'{o}n de difeomorfismos $\mathcal{C}_{a} = 0$ al pasar del fromalismo ADM \eqref{eq:constrictions} al formalismo en Variables de Conexi\'{o}n hecho por Ashtekar \eqref{eq:CaEA}, muchos de los c\'{a}lculos en el marco general de la teor\'{i}a siguen siendo poco comprendidos y bastante complejos por lo que se recurre a trabajar en marcos donde se presentan ciertas propiedades de simetr\'{i}a.

En el r\'{e}gimen cl\'{a}sico la reducci\'{o}n de simetr\'{i}a simplifica en gran medida los c\'{a}lculos englobando muy bien la din\'{a}mica a gran escala del universo, como ejemplo tenemos a la m\'{e}trica de Friedmann-Robertson-Walker que es una soluci\'{o}n exacta a las ecuaciones de Einstein que describe un universo homog\'{e}neo e isotr\'{o}pico en expansi\'{o}n. Los resultados al realizar una reducci\'{o}n de simetr\'{i}a en el formalismo de Variables de Conexi\'{o}n de modelos cosmol\'{o}gicos homogeneos e isotr\'{o}picos proporcionan un marco de trabajo sumamente \'{u}til para analizar las singularidades cl\'{a}sicas en la teor\'{i}a cu\'{a}ntica, lo que es Cosmolog\'{i}a Cu\'{a}ntica de Lazos, ya que permite realizar la reducci\'{o}n a nivel cu\'{a}ntico seleccionando estados sim\'{e}tricos.

Un modelo cl\'{a}sico reducido, dada cierta simetr\'{i}a, se explica mediante un grupo de simetr\'{i}a $S$ actuando sobre el espacio-tiempo tal que preserva la m\'{e}trica $g_{\mu \nu}$ bajo la acci\'{o}n $s^{*}(g_{\mu \nu}) = g_{\mu \nu}$ para todo $s \in S$. {\color{red}Para identificar los campos de un modelo reducido para cierto grupo de simetr\'{i}a $S$, necesitamos una base completa de todos los tensores m\'{e}tricos invariantes involucrados, as\'{i} que necesitamos una clasificaci\'{o}n de las m\'{e}tricas invariantes para estos modelos.} Sin embargo, la clasificaci\'{o}n de m\'{e}tricas invariantes no es el objetivo de este trabajo sino encontrar la forma de las Variables de Conexi\'{o}n para modelos sim\'{e}tricos dada una clasificaci\'{o}n. Adem\'{a}s, a partir de conocer la forma de las conexiones invariantes para cierto modelo sim\'{e}trico se puede encontrar la forma de m\'{e}tricas invariantes para ese mismo modelo.

Con la clasificaci\'{o}n de haces principales sim\'{e}tricos y conexiones invariantes, como se muestra en el cap\'{i}tulo \ref{chp:SymmetryReduction}, reconstruimos la forma de las conexiones invariantes para cierto grupo de simetr\'{i}a $S$, usando la clasificaci\'{o}n del haz principal dada por el haz reducido $\tilde{Q}$ \eqref{eq:ReduceBundleQ} y la clase de equivalencia $[\lambda]$ \eqref{eq:conj}, donde $\lambda$ es un mapeo que va del subgrupo de isotrop\'{i}a\footnotemark $J \leq S$ al grupo de estructura $G = \mathrm{SU}(2)$ del haz principal. De esta manera obtuvimos la forma de las conexiones invariantes $A$ y su momento can\'{o}nicamente conjugado $E$ para todos los modelos que sean simetr\'{i}cos bajo $S$.
\footnotetext{$J = \{j \in S \; \vert \; j(y) = y \; \forall \; y \in \Sigma_{t}\}$.}

En el cap\'{i}tulo \ref{chp:SymmetryReduction} investigamos la forma de las conexiones invariantes $A$ y las triadas densitizadas $E$ para modelos de Bianchi clase A, que son todos aquellos con simetr\'{i}a espacial homogenea y para los modelos de Bianchi tipo I y tipo IX que son los que presentan isotrop\'{i}a espacial. Esto es que la acci\'{o}n del grupo de simetr\'{i}a $S$ est\'{a} dada por el mapeo $s: \Sigma_{t} \longrightarrow \Sigma_{t}$, donde $\Sigma_{t}$ son las hipersuperficies espaciales de la descomposici\'{o}n 3+1 (ver cap\'{i}tulo \ref{chp:ADM}), tal que la m\'{e}trica inducida $h_{ab}$ y la curvatura $K_{ab}$ se preservan para todo $s \in S$. La tr\'{i}ada densitizada codifica la geometr\'{i}a espacial en relaci\'{o}n con la m\'{e}trica espacial mediante $\delta^{IJ} E^{a}_{I} E^{b}_{J} = h_{0} h^{ab}$, con $h_{0}$ el determinante de $h_{ab}$.

Para modelos espacialmente homog\'{e}neos, encontramos que la forma de la conexi\'{o}n homog\'{e}nea invariante $A$ y la forma de la triada densitizada $E$ est\'{a}n determinadas por sus componentes:
%
\begin{equation*}
A^{I}_{a} = \phi^{I}_{\hat{K}} \omega^{\hat{K}}_{a} V^{-1/3}_{0} \quad \mathrm{y} \quad E^{a}_{I} = \sqrt{h_{0}} p^{\hat{K}}_{I} V^{-2/3}_{0} X^{a}_{\hat{K}}
\end{equation*}
%
respectivamente. Donde los nueve par\'{a}metros $\phi^{I}_{\hat{K}}$ son arbitrarios y $p^{\hat{K}}_{I}$ tal que la estructura simpl\'{e}ctica sea: $\{\phi^{I}_{\hat{K}}, p^{\hat{K}}_{I}\} = 8 \pi \beta G_{0} \delta^{I}_{L} \delta^{\hat{J}}_{\hat{K}}$.

En cambio en los modelos con isotrop\'{i}a espacial, el espacio fase de Relatividad General se reduce a dos dimensiones, dada la libertad permitida por $\mathrm{SU}(2)$ que deja \'{u}nicamente una componente independiente en la conexi\'{o}n $A^{I}_{a}$ y una en la triada densitizada $E^{a}_{I}$, pues las dem\'{a}s componentes ya est\'{a}n determinadas por la simetr\'{i}a. Es decir, la conexi\'{o}n isotr\'{o}pica y la triada densitizada, est\'{a}n determinadas por
%
\begin{equation*}
A^{I}_{a} = \tilde{c} \delta^{I}_{a} \quad \mathrm{y} \quad E^{a}_{I} = \tilde{p} \delta^{a}_{I},
\end{equation*}
%
respectivamente.

Un ejemplo interesante de mencionar de un modelo isotr\'{o}pico es el modelo de Friedmann-Robertson-Walker, que se describe completamente mediante el factor $a(t)$ como una soluci\'{o}n a las ecuaciones de Friedmann. En en el formalismo de Variables de Conexi\'{o}n el factor de escala $a$ y sus derivadas temporales $\dot{a}$ est\'{a}n expresados a trav\'{e}s de la componente independiente $\tilde{p}$ de la triada por $|\tilde{p}| = a^{2}$ y la componente independiente de la conexi\'{o}n $\tilde{c}$ por $\tilde{c} = \tilde{\gamma} + \beta \dot{a}$ (ver \cite{Bojowald2000} para los c\'{a}lculos). Ahora, tomando las definiciones $p := V_{0}^{2/3} \tilde{p}$ y $c := V_{0}^{1/3} \tilde{c}$ (junto con $\gamma = V^{1/3}_{0} \tilde{\gamma}$) hacen que el marco de trabajo sea independiente del fondo. Un resultado importante es en la constricci\'{o}n Hamiltoniana $\mathcal{C}$, que restringida a Variables de Conexi\'{o}n isotr\'{o}picas pasa a ser \cite{Bojowald2002, Bojowald2005}
%
\begin{equation}
\mathcal{C} = -\frac{3}{8 \pi G_{0}} \left( \beta^{-2} (c - \gamma)^{2} + \gamma \right) \sqrt{|p|} + H_{\mathrm{materia}} (p) = 0.
\end{equation}
%
que sustituyendo para obtener en t\'{e}rminos de $a$ y $\dot{a}$, y reescribiendo se obtiene la ecuaci\'{o}n de Friedmann:
%
\begin{equation}
\left( \frac{\dot{a}}{a} \right)^{2} + \frac{k}{a^{2}} = \frac{8 \pi G_{0}}{3} a^{-3} H_{\mathrm{materia}}(a).
\end{equation}
%
Aunque al introducir triadas y conexiones no cambia la situaci\'{o}n en el aspecto cl\'{a}sico que existe en el espacio de m\'{e}tricas para $a = 0$. En el espacio fase c\'{a}sico de m\'{e}tricas $a = 0$ es un l\'{i}mite mientras que en el espacio fase cl\'{asico} de triadas $p = 0$ es una subvariedad lo cual en la teor\'{i}a cu\'{a}ntica tiene consecuencias importantes. {\color{red} Uno de los resultados es que al cuantizar el espacio, $p$, adem\'{a}s de poder ser negativa, toma valores discretos por lo tanto la singularidad del \emph{Big Bang} nunca ocurre, mas bien tenemos lo que se conoce como el \emph{Big Bounce} \cite{BojowaldBBB}.}
