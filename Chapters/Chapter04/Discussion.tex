La Gravedad Cu\'{a}ntica de Lazos es un candidato para la cuantizaci\'{o}n de la gravedad a partir de los fundamentos de Relatividad General, hasta ahora ha descubierto varias caracter\'{i}sticas cruciales relevantes en escalas de longitudes peque\~{n}as. Adem\'{a}s, ha logrado discretizar la geometr\'{i}a espacial a escalas de Planck, lo que ha significado en una gran ayuda en el estudio de las singularidades cosmol\'{o}gicas y de los agujeros negros. En esta tesis, investigamos la forma de las Variables de Conexi\'{o}n para modelos homogeneos e isotr\'{o}picos de cosmolog\'{i}a.

Al imponer isotrop\'{i}a espacial el espacio fase de Relatividad General se reduce a dos dimensiones, dada la libertad permitida por $\mathrm{SU}(2)$ que deja \'{u}nicamente una componente independiente en la conexi\'{o}n $A^{I}_{a}$ y una en la triada densitizada $E^{a}_{I}$, pues las dem\'{a}s componentes ya est\'{a}n determinadas por la simetr\'{i}a. Es decir, la conexi\'{o}n isotr\'{o}pica y la triada densitizada, est\'{a}n determinadas por $A^{I}_{a} = \tilde{c} \delta^{I}_{a}$ y $E^{a}_{I} = \tilde{p} \delta^{a}_{I}$, respectivamente. La tr\'{i}ada densitizada codifica la geometr\'{i}a espacial en relaci\'{o}n con la m\'{e}trica espacial mediante $|\det (E)|^{-1} \delta^{IJ} E^{a}_{I} E^{b}_{J} = h^{ab} $

Por otro lado tenemos que un modelo isotr\'{o}pico se describe completamente mediante el factor $a(t)$ como una soluci\'{o}n a las ecuaciones de Friedmann, una vez que la parte de materia ha sido especificada. En en el formalismo de Variables de Conexi\'{o}n el factor de escala $a$ y sus derivadas temporales $\dot{a}$ est\'{a}n expresados a trav\'{e}s de la componente independiente $\tilde{p}$ de la triada por $|\tilde{p}| = a^{2}$ y la componente independiente de la conexi\'{o}n $\tilde{c}$ por $\tilde{c} = \gamma \dot{a}$ (ver \cite{Bojowald2000} para los c\'{a}lculos). Ahora, tomando las definiciones $p := V_{0}^{2/3} \tilde{p}$ y $c := V_{0}^{1/3} \tilde{c}$ que hacen que el marco de trabajo independiente del fondo, el Hamiltoniano restringido a Variables de Conexi\'{o}n isotr\'{o}picas est\'{a} dado por
%
\begin{equation}
H = -\frac{3}{8 \pi G} \gamma^{-2} c^{2} \sqrt{|p|} + H_{\mathrm{materia}} (p) = 0,
\end{equation}
%
que es la ecuaci\'{o}n de Friedmann.

%Esta situaci\'{o}n, ciertamente, no cambia introduciendo tr\'{i}adas como variables en lugar de la m\'{e}trica. Sin embargo, la situaci\'{o}n ya es diferente porque p = 0 es una sub-variedad en el espacio fase cl\'{a}sico de tr\'{i}adas donde $p$ puede tener ambos signos (el signo que determina si la tr\'{i}ada es izquierda o derecha, es decir, la orientaci\'{o}n). Esto contrasta con la m\'{e}trica donde $a = 0$ es un l\'{i}mite del espacio fase cl\'{a}sico. No hay implicaciones en la teoría clásica ya que las trayectorias terminan allí, pero tendrá ramificaciones importantes en la teoría cuántica.