La Gravedad Cu\'{a}ntica de Lazos es un candidato para la cuantizaci\'{o}n de la gravedad a partir de los fundamentos de Relatividad General, hasta ahora ha descubierto varias caracter\'{i}sticas cruciales relevantes en escalas de la longitud de Planck. Adem\'{a}s, ha logrado discretizar la geometr\'{i}a espacial en cu\'{a}ntos geom\'{e}tricos de longitud, \'{a}rea y volumen, lo que ha significado una gran ayuda en el estudio de las singularidades cosmol\'{o}gicas y de los agujeros negros. Sin embargo, a pesar de simplificar la contricci\'{o}n de difeomorfismos al pasar del fromalismo ADM \eqref{eq:constrictions} al formalismo en Variables de Conexi\'{o}n hecho por Ashtekar \eqref{eq:CaEA}, muchos de los c\'{a}lculos en el marco general de la teor\'{i}a siguen siendo poco comprendidos y bastante complejos.

En el r\'{e}gimen cl\'{a}sico la reducci\'{o}n de simetr\'{i}a simplifica en gran medida los c\'{a}lculos de la ecuaciones de Einstein englobando muy bien la din\'{a}mica a gran escala del universo, como ejemplo tenemos a la m\'{e}trico de de Friedmann-Robertson-Walker. En el cap\'{i}tulo \ref{chp:SymmetryReduction} relizamos una reducci\'{o}n de simetr\'{i}a para encontrar la forma de las Variables de Conexi\'{o}n de modelos cosmol\'{o}gicos homogeneos e isotr\'{o}picos. Estos resultados proporcionan un marco de trabajo sumamente \'{u}til para analizar las singularidades cl\'{a}sicas en la teor\'{i}a cu\'{a}ntica, lo que es Cosmolog\'{i}a Cu\'{a}ntica de Lazos, ya que esto permite realizar la reducci\'{o}n de simetr\'{i}a a nivel cu\'{a}ntico seleccionando estados sim\'{e}tricos. {\color{red} En la representaci\'{o}n de conexiones, estos estados por definici\'{o}n est\'{a}n soportados s\'{o}lo por conexiones invariantes con respecto a la acci\'{o}n dada del grupo de simetr\'{i}a \cite{Kastrup, Bojowald2002}.}

{\color{OliveGreen} Hablar de la reducci\'{o}n de simetr\'{i}a, m\'{e}trica invariante, reconstrucci\'{o}n de conexi\'{o}n invariante, clasificaci\'{o}n y relaci\'{o}n con la m\'{e}trica. La tr\'{i}ada densitizada codifica la geometr\'{i}a espacial en relaci\'{o}n con la m\'{e}trica espacial mediante $|\det (E)|^{-1} \delta^{IJ} E^{a}_{I} E^{b}_{J} = h^{ab} $}

Para modelos de Bianchi clase A, que son todos los modelos homog\'{e}neos, encontramos que la forma de la conexi\'{o}n homog\'{e}nea invariante $A$ y la forma de la triada densitizada $E$ est\'{a}n determinadas por sus componentes:
%
\begin{equation*}
A^{I}_{a} = \phi^{I}_{\hat{K}} \omega^{\hat{K}}_{a} V^{-1/3}_{0} \quad \mathrm{y} \quad E^{a}_{I} = \sqrt{h_{0}} p^{\hat{K}}_{I} V^{-2/3}_{0} X^{a}_{\hat{K}}
\end{equation*}
%
respectivamente. Donde los nueve par\'{a}metros $\phi^{I}_{\hat{K}}$ son arbitrarios y $p^{\hat{K}}_{I}$ tal que se satisfaga $\{\phi^{I}_{\hat{K}}, p^{\hat{K}}_{I}\} = 8 \pi \beta G_{0} \delta^{I}_{L} \delta^{\hat{J}}_{\hat{K}}$.

En cambio en los modelos de Bianchi tipo IX, donde se impone isotrop\'{i}a espacial, el espacio fase de Relatividad General se reduce a dos dimensiones, dada la libertad permitida por $\mathrm{SU}(2)$ que deja \'{u}nicamente una componente independiente en la conexi\'{o}n $A^{I}_{a}$ y una en la triada densitizada $E^{a}_{I}$, pues las dem\'{a}s componentes ya est\'{a}n determinadas por la simetr\'{i}a. Es decir, la conexi\'{o}n isotr\'{o}pica y la triada densitizada, est\'{a}n determinadas por
%
\begin{equation*}
A^{I}_{a} = \tilde{c} \delta^{I}_{a} \quad \mathrm{y} \quad E^{a}_{I} = \tilde{p} \delta^{a}_{I},
\end{equation*}
%
respectivamente. Por otro lado tenemos que un modelo isotr\'{o}pico, por ejemplo el modelo de Friedmann-Robertson-Walker, se describe completamente mediante el factor $a(t)$ como una soluci\'{o}n a las ecuaciones de Friedmann, lo que deja un \'{u}nico grado de libertad. En en el formalismo de Variables de Conexi\'{o}n el factor de escala $a$ y sus derivadas temporales $\dot{a}$ est\'{a}n expresados a trav\'{e}s de la componente independiente $\tilde{p}$ de la triada por $|\tilde{p}| = a^{2}$ y la componente independiente de la conexi\'{o}n $\tilde{c}$ por $\tilde{c} = \tilde{\gamma} + \beta \dot{a}$ (ver \cite{Bojowald2000} para los c\'{a}lculos). Ahora, tomando las definiciones $p := V_{0}^{2/3} \tilde{p}$ y $c := V_{0}^{1/3} \tilde{c}$ (junto con $\gamma = V^{1/3}_{0} \tilde{\gamma}$) hacen que el marco de trabajo sea independiente del fondo. Un resultado importante es en la constricci\'{o}n Hamiltoniana $\mathcal{C}$, que restringida a Variables de Conexi\'{o}n isotr\'{o}picas pasa a ser \cite{Bojowald2002, Bojowald2005}
%
\begin{equation}
\mathcal{C} = -\frac{3}{8 \pi G_{0}} \left( \beta^{-2} (c - \gamma)^{2} + \gamma \right) \sqrt{|p|} + H_{\mathrm{materia}} (p) = 0.
\end{equation}
%
que sustituyendo para obtener en t\'{e}rminos de $a$ y $\dot{a}$, y reescribiendo se obtiene la ecuaci\'{o}n de Friedmann:
%
\begin{equation}
\left( \frac{\dot{a}}{a} \right)^{2} + \frac{k}{a^{2}} = \frac{8 \pi G_{0}}{3} a^{-3} H_{\mathrm{materia}}(a).
\end{equation}
%
Aunque al introducir triadas y conexiones no cambia la situaci\'{o}n en el aspecto cl\'{a}sico que existe en el espacio de m\'{e}tricas para $a = 0$. En el espacio fase c\'{a}sico de m\'{e}tricas $a = 0$ es un l\'{i}mite mientras que en el espacio fase cl\'{asico} de triadas $p = 0$ es una subvariedad lo cual en la teor\'{i}a cu\'{a}ntica tiene consecuencias importantes. {\color{red} Uno de los resultados es que al cuantizar el espacio, $p$ toma valores discretos y la singularidad del \emph{Big Bang} nunca ocurre, mas bien tenemos lo que se conoce como el \emph{Big Bounce} \cite{BB}.}