%---------------------------------------------------------------------------
% TRIADAS Y TETRADAS
%---------------------------------------------------------------------------

Consideremos una variedad diferenciable $\mathcal{M}$ $n$-dimensional y orientada. Requerimos que $\mathcal{M}$ sea paracompacta, conectada y abierta. El requisito de que la variedad sea paracompacta asegura que siempre admita una m\'{e}trica con signatura Lorentziana, mientras que los otros son por consideraciones f\'{i}sicas.

Como $\mathcal{M}$ es difeomorfa a $\mathbb{R}^n$ y el haz tangente de $\mathbb{R}^{n}$ es trivial, entonces el haz $T \mathcal{M}$ tambi\'{e}n lo es. Una trivilizaci\'{o}n es un isomorfismo de haces, por lo tanto una trivilizaci\'{o}n de $T \mathcal{M}$ es un isomorfismo
%
\begin{equation}
\label{eq:framefield}
e: \mathcal{M} \times \mathbb{R}^{n} \rightarrow T \mathcal{M}
\end{equation}
%
tal que a cada fibra $\{p\} \times \mathbb{R}^{n}$ del haz trivial $\mathcal{M} \times \mathbb{R}^{n}$ la manda al espacio tangente $T_{p} \mathcal{M}$ correspondiente, con $p \in \mathcal{M}$. El inverso de $e$ es:
%
\begin{equation}
\label{eq:invframefield}
e^{-1}: T \mathcal{M} \rightarrow \mathcal{M} \times \mathbb{R}^{n}.
\end{equation}

A la trivilizaci\'{o}n $e$ \eqref{eq:framefield} de $T \mathcal{M}$ se le conoce como \emph{campo de marcos}. Si $\mathcal{M}$ es tridimensional o cuatro-dimensional, entonces $e$ se llama \emph{triada} o \emph{t\'{e}trada} sobre $\mathcal{M}$ respectivamente.

Supongamos que estamos en el caso Lorentziano. Una secci\'{o}n de $\mathcal{M} \times \mathbb{R}^{n}$ es una funci\'{o}n $\mathbb{R}^{n}$-valuada sobre $\mathcal{M}$, por lo que la base natural de secciones es
%
\begin{align*}
\xi_{0}(p) & = (1, 0,..., 0), \\
\xi_{1}(p) & = (0, 1,..., 0),\\
& \vdots \\
\xi_{n-1}(p) & = (0, 0,..., 1).
\end{align*}
%
De este modo, cualquier secci\'{o}n $s \in \Gamma (\mathcal{M} \times  \mathbb{R}^{n})$ se puede escribir como $s^{I} \xi_{I}$. A $\mathbb{R}^{n}$, que es la fibra del haz, le llamaremos \emph{espacio interno}. As\'{i}, las letras may\'{u}sculas lat\'{i}nas, $I, \, J, \, ...$, denotan \emph{\'{i}ndices internos} asociados a la base de secciones $\xi_{I}$. Y, usaremos letras griegas, $\mu, \, \nu, \, ...$, para denotar los \'{i}ndices espacio-temporales de $M$ asociados a una base coordenada $\partial_{\mu}$ dada por una carta en $\mathcal{M}$.

El campo de marcos \footnotemark $e: \mathcal{M} \times \mathbb{R}^{n} \rightarrow T \mathcal{M}$, definen un mapeo de secciones del espacio total $\mathcal{M} \times \mathbb{R}^{n}$ a campos vectoriales en $\mathcal{M}$, que denotaremos tambi\'{e}n con $e$,
\footnotetext{Para el caso  de las triadas el espacio interno es $\mathbb{R}^{3}$ y para tetradas $\mathbb{R}^{4}$.}
%
\begin{equation}
e: \Gamma(\mathcal{M} \times \mathbb{R}^{4}) \rightarrow \Gamma(T \mathcal{M}).
\end{equation}
%
Aplicando este mapeo a las secciones $\xi_{I}$, obtenemos una base de campos vectoriales $\{e(\xi_{I})\}$ sobre $\mathcal{M}$, y en una carta podemos escribir esto como
%
\begin{equation}
\label{eq:ed}
e(\xi_{I}) = e^{\mu}_{I} \partial_{\mu},
\end{equation}
%
donde las componentes $e^{\mu}_{I}$ son componentes sobre $\mathcal{M}$. Es usual denotar $e(\xi_{I})$ como $e_{I}$, y dado que los campos vectoriales $e_{I}$ o las componentes $e^{\mu}_{I}$ son suficientes para determinar el campo de marcos es com\'{u}n llamar a cualquiera de ellos campo de marcos.

Ahora bien, en el espacio total $\mathcal{M} \times \mathbb{R}^{n}$ hay un producto interno can\'{o}nico bien definido. En otras palabras, dadas dos secciones $s, s' \in \Gamma(\mathcal{M} \times \mathbb{R}^{n})$, su producto interno es
%
\begin{equation}
\eta(s,s') = \eta_{IJ}s^{I}s'^{J}.
\end{equation}

$\eta$ es la m\'{e}trica del espacio interno $\mathbb{R}^{n}$ as\'{i} que se le llama \emph{m\'{e}trica interna} y
%
\[
\eta_{IJ} =
\begin{pmatrix}
-1 & 0 & ... & 0 \\
0 & 1 & ... & 0 \\
\vdots & \vdots & \ddots & \vdots \\
0 & 0 & ... & 1
\end{pmatrix}
\] se saca de la m\'{e}trica de Minkowski.

Usando $\eta_{IJ}$ o su inversa $\eta^{IJ}$, podemos subir o bajar \'{i}ndices internos, de esta manera mapeamos elementos del espacio interno $\mathbb{R}^{n}$ al espacio interno dual, o vice versa.

Supongamos que $\mathcal{M}$ tiene una m\'{e}trica (lorentziana) $g$, entonces el producto interno de campos vectoriales sobre $\mathcal{M}$ es $g(v, v') = g_{\mu \nu} v^{\mu} v'^{\nu}$. Con lo que si el producto interno de los campos vectoriales
%
\begin{equation}
\label{eq:g(ee)}
g(e_{I}, e_{J}) = \eta_{IJ},
\end{equation}
%
es decir, $\{e_{I}\}$ es ortonormal, decimos que el campo de marcos es ortonormal.

Si el campo de marcos es ortonormal, la m\'{e}trica $g$ en $\mathcal{M}$ est\'{a} bien relacionada a la m\'{e}trica interna $\eta$:
%
\begin{align*}
g(e(s), e(s')) & = g(e(s^{I} \xi_{I}), e(s^{J} \xi_{J})) = s^{I} s^{J} g(e_{I}, e_{J}) \\
& = \eta_{IJ} s^{I} s^{J} = \eta(s^{I} \xi_{I}, s^{J} \xi_{J}) \\
& = \eta(s,s')
\end{align*}
%
\begin{equation}
\therefore g(e(s), e(s')) = \eta(s,s').
\end{equation}
%
Para cualesquiera secciones $s$, $s'$ de $\mathcal{M} \times \mathbb{R}^{n}$. Este resultado implica que la m\'{e}trica en $\mathcal{M}$ est\'{a} dada en t\'{e}rminos del campo de marcos inverso,
%
\begin{equation}
g(v, v') = \eta(e^{-1}(v), e^{-1}(v')).
\end{equation}
%
Ahora, por \eqref{eq:g(ee)} y \eqref{eq:ed} tenemos que
%
\begin{equation}
\eta_{IJ} = g_{\mu \nu} e^{\mu}_{I} e^{\nu}_{J},
\end{equation}
%
entonces
%
\begin{equation}
\delta^{I}_{J} = e^{I}_{\alpha} e^{\alpha}_{J}.
\end{equation}

A $e^{I}_{\alpha}$ se le denomina \emph{co-campo de marcos}\footnotemark.
\footnotetext{\emph{Cotriada} si la dimensi\'{o}n es tres o \emph{cot\'{e}trada} si es cuatro-dimensional.}

Dada una secci\'{o}n $s \in \Gamma(\mathcal{M} \times \mathbb{R}^{n})$ tenemos que $e(s) = v$ donde $v \in \Gamma(T \mathcal{M})$, entonces $v^{\alpha} \partial_{\alpha} = s^{I} e^{\alpha}_{I} \partial_{\alpha}$
%
\begin{equation}
\label{eq:sieai}
\Longrightarrow \, v^{\alpha} = s^{I}e^{\alpha}_{I}.
\end{equation}
%
Y si ahora contraemos esta igualdad \eqref{eq:sieai} con $e^{J}_{\alpha}$ obtenemos que
%
\begin{equation}
s^{J} = e^{J}_{\alpha} v^{\alpha}.
\end{equation}

Por otro lado, como $e^{-1}(v) = s$ entonces $s^{I} \xi_{I} = e^{I}_{\alpha} v^{\alpha} \xi_{I} = v^{\alpha} e^{-1}(\partial_{\alpha})$
%
\begin{equation}
\label{eq:e-1aeixii}
\Longrightarrow \, e^{-1}(\partial_{\alpha}) = e^{I}_{\alpha} \xi_{I}.
\end{equation}

Utilizando est\'{a}s \'{u}ltimas relaciones, \eqref{eq:sieai} y \eqref{eq:e-1aeixii}, podemos demostrar que las componentes de la m\'{e}trica $g$ se expresan como:
%
\begin{equation}
g_{\mu \nu} = \eta_{IJ} e^{I}_{\mu} e^{J}_{\nu}.
\end{equation}

Relaci\'{o}n de la cual se sigue
%
\begin{equation}
\delta^{\alpha}_{\beta} = e_{I}^{\alpha} e^{J}_{\beta}.
\end{equation}