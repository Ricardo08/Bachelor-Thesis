%----------------------------------------------------------------------------------------
% FIBER BUNDLES
%----------------------------------------------------------------------------------------

\begin{mydef}[Haz fibrado] \cite{Baez}
Un \emph{haz fibrado} es una estructura $(E, \pi, \mathcal{M})$, donde $E$ y $\mathcal{M}$ son variedades, y $\pi: E \longrightarrow \mathcal{M}$ es un mapeo sobreyectivo continuo. Las variedades $E$ y $\mathcal{M}$ son llamadas \emph{espacio total} y \emph{espacio base} respectivamente, mientras que $\pi$ se le conoce como \emph{mapeo de proyecci\'{o}n} o \emph{proyecci\'{o}n de haz}.
%
\begin{center}
\begin{tikzpicture}[node distance=2.0cm,auto]
\node (A) {$E$};
\node (B) [right=of A] {$\mathcal{M}$.};

\path[->, >=stealth] (A) edge node {$\pi$} (B);
\end{tikzpicture}
\end{center}
%
\end{mydef}

El nombre haz fibrado viene de que el espacio total $E$ es la uni\'{o}n de variedades $F_{m}$, $$E = \bigcup\limits_{m \in \mathcal{M}} F_{m},$$ donde a $F_{m}$ se le conoce como la fibra de $m$ y es el espacio $F_{m} = \{p \in E \; \vert \; \pi(p) = m, \; m \in \mathcal{M}\}$.

Un haz fibrado requiere que el mapa de proyecci\'{o}n $\pi$ para cada punto $m \in \mathcal{M}$ tenga una vecindad abierta $U$ tal que exista un homeomorfismo $\varphi: \pi^{-1}(U) \longrightarrow U \times F$, es decir, $\pi$ debe coincidir con la proyecci\'{o}n sobre el primer factor al espacio base, i.e. $\mathrm{proj_{1}} \circ \varphi = \pi$. Esto es que el siguiente diagrama conmute,
%
\begin{center}
\begin{tikzpicture}[node distance=2.0cm,auto]
\node (A) {$\pi^{-1}(U) \subset E$};
\node (B) [right=of A] {$U \times F$};
\node (C) [below=of A] {$U \subset \mathcal{M}$};

\path[->, >=stealth] (A) edge node {$\varphi$} (B);
\path[->, >=stealth] (A) edge node {$\pi$} (C);
\path[->, >=stealth] (B) edge node {$\mathrm{proj_{1}}$} (C);
\end{tikzpicture}
\end{center}

Los homeomorfismos $\varphi$ que ``conmutan con la proyecci\'{o}n'' son llamados \emph{trivilizaci\'{o}n local} del haz fibrado. En otras palabras $E$ se ve como el producto cartesiano $\mathcal{M} \times F$ al menos localmente. Y se dice que el mapa de proyecci\'{o}n $\pi$ es \emph{localmente trivial}.

\begin{mydef}[Carta, Atlas]  \cite{Baez}
Al par $(U, \varphi)$ se le llama \emph{carta de la fibra} y es una trivilizaci\'{o}n local. El conjunto de todas ellas $\{(U_{i}, \varphi_{i})\}$ se le conoce como \emph{atlas del haz fibrado}.
\end{mydef}

\begin{mydef}[Haz trivial]  \cite{Baez}
Sea $E=\mathcal{M} \times F$, con fibra $F$ y mapeo de proyecci\'{o}n sobre el primer factor, i.e. $\pi = \mathrm{proj}_{1}$, entonces el haz $(\mathcal{M} \times F, \mathrm{proj}_{1}, \mathcal{M})$ es llamado \emph{haz trivial}.
%
\begin{align*}
\pi: \mathcal{M} \times F & \longrightarrow \mathcal{M}, \\
(m, p) & \longmapsto m.
\end{align*}
%
\end{mydef}

\begin{mydef}[Restricci\'{o}n]  \cite{Baez}
Dada la subvariedad $\mathcal{N} \subseteq \mathcal{M}$, la \emph{restricci\'{o}n} a $\mathcal{N}$ tiene como espacio total a $E\vert_{\mathcal{N}} = \{p \in E \, \vert \; \pi(p) \in \mathcal{N}\}$, a $\mathcal{N}$ como espacio base y el mapeo de proyecci\'{o}n $\pi$ restringido a $E\vert_{\mathcal{N}}$.
\end{mydef}

\begin{mydef}[Subhaz fibrado]  \cite{Baez}
Un haz $(E', \pi', \mathcal{M}')$ es un \emph{subhaz} de $(E, \pi, \mathcal{M})$ siempre que $E'$ sea un subespacio de $E$, $\mathcal{M}'$ sea subespacio de $\mathcal{M}$, y $\pi' = \pi\vert_{E'}$ tal que,
%
\begin{center}
\begin{tikzpicture}[node distance=2.0cm,auto]
\node (A) {$E'$};
\node (B) [right=of A] {$\mathcal{M'}$.};

\path[->, >=stealth] (A) edge node {$\pi'$} (B);
\end{tikzpicture}
\end{center}
%
\end{mydef}

\begin{ex}
El \emph{haz tangente} de una variedad diferenciable $\mathcal{M}$, es aquel cuyo espacio total es la uni\'{o}n de los espacios tangentes a cada uno de sus puntos, i.e. $E =  \bigcup\limits_{m \in \mathcal{M}} T_{m} \mathcal{M} := T \mathcal{M}$. La proyecci\'{o}n $\pi: T \mathcal{M} \longrightarrow \mathcal{M}$ mapea a cada vector tangente $v \in T_{m} \mathcal{M}$ a $m \in \mathcal{M}$. La fibra en cada $m \in \mathcal{M}$ es el espacio tangente $T_{m} \mathcal{M}$.
\end{ex}

\begin{mydef}[Haz Vectorial]  \cite{Baez}
Un \emph{haz vectorial}, $n$-dimensional, es un haz $(E, \pi, \mathcal{M})$ cuya fibra $\pi^{-1}(m)$ con $m \in \mathcal{M}$ tiene estructura de un espacio vectorial ($n$-dimensional). Adem\'{a}s, en cada punto $m \in \mathcal{M}$ se requiere que exista una vecindad abierta $U_{m}$ y una trivilizaci\'{o}n $\varphi: E \vert_{U} \longrightarrow U \times \mathbb{R}^{n}$ que mande cada fibra $\pi^{-1}(m)$ a la fibra $\{m\} \times \mathbb{R}^{n}$ linealmente. Esto es equivalente a pedir que la trivilizaci\'{o}n sea lineal en las fibras.
\end{mydef}

%----------------------------------------------------------------------------------------
% SECTIONS
%----------------------------------------------------------------------------------------

\begin{mydef}[Secci\'{o}n]  \cite{Baez}
Sea $(E, \pi, \mathcal{M})$ un haz. Un mapeo $$\sigma: \mathcal{M} \longrightarrow E,$$ es llamado una secci\'{o}n del haz si $\pi \circ \sigma = \mathrm{id}_{\mathcal{M}}$.
%
\begin{center}
\begin{tikzpicture}
% the bottom left border of the surface
\path[name path=border1] (-3,-0.5) to[out=-10,in=150] (3,-0.5);
% the upper right border of the surface
\path[name path=border2] (2,0.5) to[out=150,in=-10] (-4, 0.5);
% we draw the surface
\draw (-3, -0.5) to[out=-10,in=150] (3,-0.5) -- (2,0.5) to[out=150,in=-10] (-4, 0.5) -- cycle;
% we draw the curved black line on top
\draw (-3.5, 2.2) to[out=-20,in=220] 
  coordinate[pos=0.3] (aux1) 
  coordinate[pos=0.7] (aux2)(1.8, 3.2);
% we draw the filled black circles
\coordinate (bux1) at (-1.6,0); 
\coordinate (bux2) at (0.6,0.4);
\draw[fill] (bux1) circle (1pt);
\draw[fill] (bux2) circle (1pt);
% we draw the markers and place the labels
\foreach \coor/\subs in {1/p,2/q}
{
  \draw[fill] (aux\coor) circle (1pt);
  \node[label=above:$\subs$] at (aux\coor) {};
  \path[<-, bend left] (aux\coor) edge node {$\qquad \sigma$} (bux\coor);
  \node[label=right:$\pi(\subs)$] at (bux\coor) {};
}
%
\node[rotate=0] at (-2.6,-0.3) {$\mathcal{M}$};
\node[rotate=30] at (1.7,3.4) {$E$};
%
\end{tikzpicture}
\end{center}
%
Es decir, una secci\'{o}n es un mapeo del espacio base $\mathcal{M}$ al espacio total $E$ tal que $\sigma(m) \in \pi^{-1}(m)$, la fibra sobre $m$, para cada $m \in \mathcal{M}$.
\end{mydef}

%----------------------------------------------------------------------------------------
% MORPHISMS
%----------------------------------------------------------------------------------------

\begin{mydef}[Morfismo]
Sean $(E, \pi, \mathcal{M})$ y $(E', \pi, \mathcal{M}')$ dos haces, un \emph{morfismo} es un par de mapeos $(u, f)$, donde
%
\begin{align*}
u: E & \longrightarrow E' \\
f: \mathcal{M} & \longrightarrow \mathcal{M}',
\end{align*}
%
tal que el siguiente diagrama conmute, i.e. $\pi' \circ u = f \circ \pi$,
%
\begin{center}
\begin{tikzpicture}[node distance=2.0cm,auto]
\node (A) {$E$};
\node (B) [right=of A] {$E'$};
\node (C) [below=of A] {$\mathcal{M}$};
\node (D) [below=of B] {$\mathcal{M}'$};

\path[->, >=stealth] (A) edge node {$u$} (B);
\path[->, >=stealth] (A) edge node {$\pi$} (C);
\path[->, >=stealth] (B) edge node {$\pi'$} (D);
\path[->, >=stealth] (C) edge node {$f$} (D);
\end{tikzpicture}
\end{center}
%
\end{mydef}

\begin{obs}
La condici\'{o}n $\pi' \circ u = f \circ \pi$ tambi\'{e}n puede ser expresada por la relaci\'{o}n $u(\pi^{-1}(m)) \subseteq \pi'^{-1}(f(m))$ para cada $m \in \mathcal{M}$, es decir, la fibra $F_{m}$ es llevada a la fibra $F_{f(m)}$ por $u$.
\end{obs}

\begin{obs}
Debemos notar que $u: E \longrightarrow E'$ determina un\'{i}vocamente a $f: \mathcal{M} \longrightarrow \mathcal{M}'$, por lo que a partir de ahora cuando digamos el morfismo de haz $u$ realmente nos referimos al par  $(u, f)$.
\end{obs}

\begin{ex}
Consideremos el mapeo \emph{pushforward} de vectores tangentes $h_{\ast}: T_{m}\mathcal{M} \longrightarrow T_{h(m)}\mathcal{M}'$, con $h: \mathcal{M} \longrightarrow \mathcal{M}'$ un mapeo suave entre variedades. Ahora, $T_{m}\mathcal{M}$ es la fibra del haz tangente con espacio total $T_{m}\mathcal{M}$ y espacio base $\mathcal{M}$, de manera similar para $T_{h(m)}\mathcal{M}'$. Entonces tenemos que $$h_{\ast}: T \mathcal{M} \longrightarrow T \mathcal{M}',$$ es un morfismo de haces.
\end{ex}

\begin{mydef}[Endomorfismo]  \cite{Baez}
Un \emph{endomorfismo} es un morfismo que va de un haz $(E, \pi, \mathcal{M})$ en s\'{i} mismo.
\end{mydef}

\begin{mydef}[Isomorfismo]
Dos haces $(E, \pi, \mathcal{M})$ y $(E', \pi, \mathcal{M}')$ son \emph{isomorfos} como haces, es decir, tienen la misma estructura, si existen los morfismos de haz $(u, f)$ y $(u^{-1}, f^{-1})$. Esto es,
%
\begin{center}
\begin{tikzpicture}[node distance=2.2cm, auto]
\node (A) {$E$};
\node (B) [right=of A] {$E'$};
\node (C) [below=of A] {$\mathcal{M}$};
\node (D) [below=of B] {$\mathcal{M}'$};

\path (A.east) -- (A.north east) coordinate[pos=0.15] (A1);
\path (B.west) -- (B.north west) coordinate[pos=0.15] (B1);
\draw[->, >=stealth] (A1) -- (B1);
\path (A.east) -- (A.south east) coordinate[pos=0.15] (A2);
\path (B.west) -- (B.south west) coordinate[pos=0.15] (B2);
\draw[<-, >=stealth] (A2) -- (B2);

\path (C.east) -- (C.north east) coordinate[pos=0.15] (C1);
\path (D.west) -- (D.north west) coordinate[pos=0.15] (D1);
\draw[->, >=stealth] (C1) -- (D1);
\path (C.east) -- (C.south east) coordinate[pos=0.15] (C2);
\path (D.west) -- (D.south west) coordinate[pos=0.15] (D2);
\draw[<-, >=stealth] (C2) -- (D2);

\path[->, >=stealth] (A) edge node {$\pi$} (C);
\path[->, >=stealth] (B) edge node {$\pi'$} (D);

\draw (1.3, 0.3) node(b) {$u$};
\draw (1.4, -0.3) node(b) {$u^{-1}$};
\draw (1.3, -2.4) node(b) {$f$};
\draw (1.4, -3.1) node(b) {$f^{-1}$};
%
\end{tikzpicture}
\end{center}
%
Tal $(u, f)$ es llamado \emph{isomorfismo de haz}.
\end{mydef}

\begin{mydef}[Automorfismo]
Se dice que $u$ es un \emph{automorfismo} si es un endomorfismo que tambi\'{e}n es un isomorfismo.
\end{mydef}

\begin{mydef}[Haz Principal]  \cite{Kobayashi}
Un haz $(P, \pi, \mathcal{M})$ es llamado \emph{haz principal} o \emph{$G$-haz principal} ($G$ especif\'{i}ca el grupo de Lie) si
\begin{itemize}
\item{$P$ es un $G$-espacio derecho, es decir, $G$ act\'{u}a por la derecha sobre $P$.}
\item{La acci\'{o}n de $G$ es libre.}
\item{Y si cada fibra $\pi^{-1}(m)$, para todo $m \in \mathcal{M}$, es isomorfa a $G$.}
\end{itemize}
Al grupo $G$ se le conoce como el grupo de estructura del haz principal $(P, \pi, \mathcal{M})$.
\end{mydef}

Burdamente hablando, un \emph{haz principal} o \emph{G-haz principal}, es un haz cuya fibra es un grupo de Lie.