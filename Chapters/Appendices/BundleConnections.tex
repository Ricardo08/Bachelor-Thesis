\section{Conexiones}

Una conexi\'{o}n en un haz fibrado es el modo de definir la noci\'{o}n de transporte paralelo en el haz; esto es, una manera de conectar o identificar fibras sobre puntos cercanos.

Sea $\pi:E \rightarrow \mathcal{M}$ un haz vectorial suave sobre una variedad diferenciable $\mathcal{M}$. Denotemos el espacio de secciones (suaves) de $E$ por $\Gamma (E)$. Una \emph{conexi\'{o}n} en $E$ es un objeto que asigna un campo vectorial $X \in \mathcal{X(M)}$ ($\mathcal{X(M)}$ denota el conjunto de campos vectoriales en $\mathcal{M}$) una funci\'{o}n
%
\begin{equation}
D: \Gamma (E) \rightarrow \Gamma(E),
\end{equation}
%
tal que satisface las siguientes propiedades
%
\begin{align*}
D_{X} (\alpha s_{1} + s_{2}) & = \alpha D_{X} (s_{1}) + D_{X} (s_{2}), \\
D_{X_{1} + X_{2}} (s) & = D_{X_{1}} (s) + D_{X_{2}} (s), \\
D_{X} (fs) & = f D_{X} (s) + X[f] s, \\
D_{fX} (s) & = f D_{X} (s),
\end{align*}
para todo: $s \in \Gamma(E)$, $X \in \mathcal{X(M)}$, $f \in C^{\infty}(\mathcal{M})$ y escalar\footnotemark $\alpha$.
\footnotetext{Aqu\'{i} escalar puede ser real o complejo depender\'{a} si el haz es real o complejo.}

En particular, la tercer propiedad, la regla de Leibniz, es lo que hace a $D_{X}$ actuar como diferenciaci\'{o}n. Dada cualquier secci\'{o}n $s$ y cualquier campo vectorial $X$, a $D_{X} (s)$ se le llama \emph{derivada covariante} de $s$ en la direcci\'{o}n de $X$.

Consideremos un sistema de coordenadas local $\{x^{\mu}\}$ en un abierto $U \subseteq \mathcal{M}$, con la correspondiente base $\{\partial_{\mu}\}$ de los campos vectoriales coordenados y una base local de secciones $\{e_{I}\}$ del haz sobre $U$. Denotamos la derivada covariante en direcci\'{o}n $\partial_{\mu}$ como $D_{\partial_{\mu}} = D_{\mu}$. De esta manera

\begin{align}
\label{eq:nabla}
D_{X}(s) & = X[s^{I}] e_{I} + s^{I} D_{X} (e_{I}) \nonumber \\
& = x^{\mu} \partial_{\mu} [s^{I}] e_{I} + s^{I} D_{x^{\mu} \partial_{\mu}} (e_{I}) \nonumber \\
& = x^{\mu} (\partial_{\mu} (s^{I}) e_{I} + s^{I} D_{\mu} (e_{I})) \nonumber \\
& = x^{\mu} (\partial_{\mu} (s^{I}) e_{I} + s^{I} A^{J}_{\mu I} (e_{J})) \nonumber \\
& = x^{\mu} (\partial_{\mu} (s^{I}) + A^{I}_{\mu J} s^{J}) (e_{I}) \nonumber \\
& = x^{\mu} D_{\mu} (s^{I}) e_{I},
\end{align}
%
donde $D_{X}(s)$ queda totalmente definido por $A^{I}_{\mu J}$. Pues para cuales quiera $\mu$ y $I$, se puede expresar $D_{\mu} (e_{I})$ de manera \'{u}nica como una combinaci\'{o}n lineal de secciones $e_{I}$, con funciones $A^{I}_{\mu J}$ en $U$ como coeficientes,
%
\begin{equation}
D_{\mu} (e_{I}) = A^{J}_{\mu I} e_{J}.
\end{equation}
%
Adem\'{a}s, en \eqref{eq:nabla} definimos $D_{\mu} (s^{I}) := \partial_{\mu} (s^{I}) + A^{I}_{\mu J} s^{J}$ que es la componente $I$ de la derivada covariante de $s$ en la base $\{e_{J}\}$.

N\'{o}tese que el potencial vectorial $A^{J}_{\mu I}$ no es solamente una funci\'{o}n etiquetada por los \'{i}ndices $\mu$, $I$, $J$ sino que es una secci\'{o}n. Al calcular $D_{X} (s)$ se obtiene el t\'{e}rmino
%
\begin{equation*}
A^{I}_{\mu J} x^{\mu} s^{J} e_{I}
\end{equation*}
%
el cual es una nueva secci\'{o}n de $E$ sobre $U$. Si se multiplica $x^{\mu}$ o $s^{J}$ por una funci\'{o}n $f \in C^{\infty}(U)$, se tendr\'{a} que toda la expresi\'{o}n anterior estar\'{a} multiplicada por $f$ ya que no aparecen derivadas; por lo que la expresi\'{o}n es $C^{\infty}(U)$-lineal. As\'{i} que, el verdadero sentido del potencial vectorial $A^{I}_{\mu J}$ es operar sobre un campo vectorial y una secci\'{o}n de $E$ sobre $U$ y dar como resultado una nueva secci\'{o}n de $E$ sobre $U$ en una manera $C^{\infty}(U)$-lineal.

Ahora bien, se puede pensar al potencial vectorial como una 1-\emph{forma} valuada en el \emph{haz de endomorfismos} $E \otimes E^{*} = \mathrm{End}(E)$ sobre $U$, es decir, como una secci\'{o}n del haz $$\mathrm{End}(E|_{U}) \otimes T^{*}U.$$ La raz\'{o}n es que si definimos el potencial vectorial $A$ de la siguiente forma
%
\begin{equation}
A = A^{I}_{\mu J} (e_{I} \otimes e^{J} \otimes dx^{\mu}),
\end{equation}
%
la parte de la 1-\emph{forma} al actuar sobre cualquier campo vectorial $X$ en $U$ $$A(X) = A^{I}_{\mu J} e_{I} \otimes e^{J} \otimes dx^{\mu}(X) = A^{I}_{\mu J} x^{\mu} (e_{I} \otimes e^{J})$$ da como resultado una secci\'{o}n de $\mathrm{End}(E)$ sobre $U$. Si ahora act\'{u}a sobre una secci\'{o}n $s$ de $E$, se tiene $$A(X)s = A^{I}_{\mu J} x^{\mu} (e_{I} \otimes e^{J}) s = A^{I}_{\mu J} x^{\mu} s^{J} e_{I}$$ que es una secci\'{o}n de $E$ sobre $U$ con componentes $s'^{I} = A^{I}_{\mu J} x^{\mu} s^{J} e_{I}$.

Cuando el haz vectorial $E$ tiene una m\'{a}s estructura, las conexiones que son compatibles con esta estructura son con las que se desea trabajar. Por ejemplo, en los $G$-haces, donde $G$ es el grupo de norma de la teor\'{i}a; aqu\'{i} las conexiones m\'{a}s importantes son aquellas donde su potencial vectorial se vea localmente como una 1-\emph{forma} valuada en el \'{a}lgebra de Lie de $G$.

Sea $\pi: P \rightarrow \mathcal{M}$ un $G$-haz principal suave sobre una variedad diferenciable $\mathcal{M}$. Entonces, una \emph{$G$-conexi\'{o}n principal} sobre $P$ es una 1-\emph{forma} diferencial en $P$ con valores en el \'{a}lgebra de Lie de $G$ la cual es $G$-equivariante y reproduce los generadores del \'{a}lgebra de Lie de los campos vectoriales fundamentales en $P$.

Decimos que $D$ es una $G$-conexi\'{o}n, si en coordenadas locales las componentes $A_{\mu} \in \mathrm{End}(E)$ est\'{a}n en el \'{a}lgebra de Lie de $G$. Aunque est\'{a}n definici\'{o}n parece depender de las coordenadas locales $\{x^{\mu}\}$ usadas para definir las componentes de $\mathrm{End}(E)$, en realidad no: si se hace un cambio de coordenadas $\{x'^{\nu}\}$, se obtienen nuevas componentes $A'_{\nu}$ dadas por
%
\begin{equation*}
A'_{\nu} = \frac{\partial x'^{\mu}}{\partial x^{\nu}} A_{\mu},
\end{equation*}
%
las cuales viven en el \'{a}lgebra de Lie de $G$ si $A_{\mu}$ lo hace.

\section{Curvatura}

Consideremos un haz vectorial $E$ sobre $\mathcal{M}$ con una conexi\'{o}n $D$. Sean $X, \, Y \in \mathcal{X(M)}$, entonces definimos la \emph{curvatura} $F(X,Y)$ como el operador sobre secciones de $E$ tal que
%
\begin{equation}
F(X,Y)s = [D_{X}, D_{Y}]s - D_{[X,Y]}s.
\end{equation}

El primer t\'{e}rmino mide la falla de las derivadas covariantes de conmutar, mientras que el segundo es una correcci\'{o}n. Esta correcci\'{o}n es para cuando se tenga una conexi\'{o}n plana, la curvatura sea id\'{e}nticamente cero. Por ejemplo, en el caso de la conexi\'{o}n plana est\'{a}ndar en un haz trivial con fibra $V$, donde una secci\'{o}n es s\'{o}lo una funci\'{o}n $f:\mathcal{M} \rightarrow V$, se tiene
%
\begin{equation}
F(X,Y)s = XYf - YXf - [X,Y]f = 0.
\end{equation}
% 
As\'{i}, cuando una conexi\'{o}n tenga curvatura nula, i.e. $F(X,Y)s = 0$, para todos los campos vectoriales $X$ y $Y$ y todas las secciones $s$, se dir\'{a} que es \emph{plana}.

Una de las propiedades de $F(X,Y)$ es que es antisim\'{e}trica, $$F(X,Y) = -F(Y,X).$$ Adem\'{a}s, debido a las propiedades de $D$ y que el par\'{e}ntesis de Lie de campos vectoriales satisface $$[X,fY] = f[X,Y] + X[f]Y,$$ para todas las funciones $f$. Entonces, la curvatura es lineal sobre $C^{\infty}(\mathcal{M})$ en cada uno de los argumentos, esto es
%
\begin{equation*}
F(fX,Y)s = F(X, fY)s = F(X,Y)fs = fF(X,Y)s.
\end{equation*}
%
Y por la definici\'{o}n de $D$, $F(X,Y)$ tambi\'{e}n define un mapeo $C^{\infty}(\mathcal{M})$-lineal $\Gamma(E) \rightarrow \Gamma(E)$. De modo que $F(X,Y)$ corresponde a una secci\'{o}n de $\mathrm{End}(E)$.

En coordenadas locales $\{x^{\mu}\}$ para un abierto $U \subseteq \mathcal{M}$ las componentes de la curvatura son
%
\begin{equation}
F_{\mu \nu} = F(\partial_{\mu}, \partial_{\nu}).
\end{equation}
%
Not\'{e}mos, como $[\partial_{\mu}, \partial_{\nu}] = 0$ entonces
%
\begin{equation}
F_{\mu \nu} = [D_{\mu}, D_{\nu}].
\end{equation}
%
De esta manera, se puede escribir la curvatura para cuales quiera campos vectoriales $X, Y \in U$ como
%
\begin{equation}
F(X,Y) = x^{\mu} x^{\nu} F_{\mu \nu}.
\end{equation}

Si tambi\'{e}n consideramos sobre una base local $\{e_{I}\}$ de secciones para $E$ sobre $U$, entonces tenemos definidas las componentes del potencial vectorial en esa base, i.e. $A^{I}_{\mu J}$. Por lo tanto, la acci\'{o}n de $F_{\mu \nu}$ sobre $e_{I}$ es
%
\begin{align*}
F_{\mu \nu} e_{I} & = [D_{\mu}, D_{\nu}] e_{I} \\
& = D_{\mu} (D_{\nu} e_{I}) - D_{\nu} (D_{\mu} e_{I}) \\
& = D_{\mu} (A^{J}_{\nu I} e_{J}) - D_{\nu} (A^{J}_{\mu I} e_{J}) \\
& = (\partial_{\mu} A^{J}_{\nu I}) e_{J} + A^{K}_{\mu J} A^{J}_{\nu I} e_{K} - (\partial_{\nu} A^{J}_{\mu I}) e_{J} - A^{K}_{\nu J} A^{J}_{\mu I} e_{K}
\end{align*}
%
Renombrando \'{i}ndices,
%
\begin{equation}
\label{eq:FmneI}
F_{\mu \nu} e_{I} = (\partial_{\mu} A^{J}_{\nu I} - \partial_{\nu} A^{J}_{\mu I} + A^{J}_{\mu K} A^{K}_{\nu I} - A^{J}_{\nu K} A^{K}_{\mu I}) e_{J}
\end{equation}

Ahora, como se ha mencionado $F(X,Y)$ es una secci\'{o}n en $\mathrm{End}(E)$, as\'{i} que $F_{\mu \nu}$ corresponde a una secci\'{o}n en $\mathrm{End}(E)$. Y dado que $e_{J} \otimes e^{I}$ forman una base local para las secciones de $\mathrm{End}(E)$, entonces se puede escribir
%
\begin{equation}
F_{\mu \nu} = F^{J}_{I \mu \nu} e_{J} \otimes e^{I},
\end{equation}
%
donde $F^{J}_{I \mu \nu}$ son las componentes de la secci\'{o}n en la base considerada. Dichas componentes son las componentes de la curvatura. Claramente,
%
\begin{equation}
\label{eq:FmneI2}
F_{\mu \nu} e_{I} = F^{J}_{I \mu \nu} e_{J}.
\end{equation}
%
Entonces de \eqref{eq:FmneI} y esta \'{u}ltima ecuaci\'{o}n, \eqref{eq:FmneI2}, se obtiene que las componentes de la curvatura son
%
\begin{equation}
F^{J}_{I \mu \nu} = \partial_{\mu} A^{J}_{\nu I} - \partial_{\nu} A^{J}_{\mu I} + A^{J}_{\mu K} A^{K}_{\nu I} - A^{J}_{\nu K} A^{K}_{\mu I}.
\end{equation}